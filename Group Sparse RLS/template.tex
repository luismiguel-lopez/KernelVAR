


%%%%%%%%%%%%%% SETTINGS %%%%%%%%%%%%%%%%%%%%%%%%%%%%%%%%%%%%%%%%%%%%%%%%%%

% SET FOLLOWING VARIABLE TO 1 FOR EDIT MODE AND 0 FOR VIEW MODE
% (erase .bbl and .aux files in this folder every time
% you switch from one mode to another, otherwise you get
% an error)
% One of the features of the edit mode is that it shows citations in
% boldface if they are not defined in the bibtex file, like the [?] in
% view mode. Different from the view mode, the edit mode indicates
% which is the missing reference.
\def\editmode{1}
\def\reportmode{1}

% set the following variable to the filenames, separated by commas, of
% the bibfiles that contain your references.  
\def\bibfilenames{WISENET}

% Use the following lines if you want to modify how citations are displayed
%% \let\oldcite\cite
%% \renewcommand{\cite}[1]{\textbf{\oldcite{#1}}}


%%%%%%%%%%%%%% BEGIN: DO NOT TOUCH REGION %%%%%%%%%%%%%%%%%%%%%%%%%%%%%%%%


\if\reportmode1
%\documentclass[journal]{IEEEtran}
%\documentclass[draftclsnofoot,onecolumn,12pt]{IEEEtran}
\documentclass[10pt,final,onecolumn]{IEEEtran}
%\documentclass[10pt,final,twocolumn]{IEEEtran}
%\documentclass[twocolumn,twoside]{IEEEtran}
\else
\documentclass[journal]{IEEEtran}
\fi

\if\editmode1  %%%%%%%%%%%%%%%%%%%%%%%%%%%%%%%%%%%%%%%%%%%%%%%%%%%%%%%%%%%
%% %% % Edit mode
\usepackage[backend=bibtex,style=alphabetic,sorting=debug]{biblatex}
\DeclareFieldFormat{labelalpha}{\thefield{entrykey}}
\DeclareFieldFormat{extraalpha}{}
\bibliography{\bibfilenames}
\newcommand{\cmt}[1]{\noindent\textcolor{lightgreen}{\underline{[#1]}}} % comment
\newcommand{\hc}[1]{\textcolor{blue}{#1}} 
% highlight command --> to
% know which symbol is
% defined as a command
\newenvironment{myitemize}{\begin{itemize}}{\end{itemize}}
\newcommand{\myitem}{\item}
\newcommand{\mycite}[1]{\textcolor{darkgreen}{[#1=\cite{#1}]}}

\else
% View mode
\usepackage{cite}
\bibliographystyle{IEEEbib}
\newcommand{\cmt}[1]{} % comment
\newcommand{\hc}[1]{\textcolor{black}{#1}} % highlight command -->
                                % in the edit mode, this command can
                                % be used to color symbols defined as
                                % commands 
\newenvironment{myitemize}{}{}
\newcommand{\myitem}{}
\newcommand{\mycite}[1]{\cite{#1}}
\fi %%%%%%%%%%%%%%%%%%%%%%%%%%%%%%%%%%%%%%%%%%%%%%%%%%%%%%%%%%%%%%%%%%%%%%

%%%%%%%%%%%%%% END: DO NOT TOUCH REGION %%%%%%%%%%%%%%%%%%%%%%%%%%%%%%%%%%

%%
%%%%%%%%%%%%%%%%%%%%%%%%%%%%%%%%%theorem environments
%
\newtheorem{assumption}{\hspace{0pt}\bf AS\hspace{-0.15cm}}

\newcounter{lemma} \setcounter{lemma}{1}
\newenvironment{lemma}{\vspace{0.1cm} \noindent {\bf Lemma
\thelemma :} \addtocounter{lemma}{-1}\refstepcounter{lemma} \em} {\addtocounter{lemma}{1}
 \normalfont }


\newcounter{proposition} \setcounter{proposition}{1}
\newenvironment{proposition}{\vspace{0.1cm} \noindent {\bf Proposition
\theproposition :} \addtocounter{proposition}{-1}\refstepcounter{proposition}\em } { \addtocounter{proposition}{1}
 \normalfont }

\newcounter{corollary} \setcounter{corollary}{1}
\newenvironment{corollary}{\vspace{0.1cm} \noindent {\bf Corollary
\thecorollary :} \addtocounter{corollary}{-1}\refstepcounter{corollary}\em } { \addtocounter{corollary}{1}
 \normalfont }




\newtheorem{observation}{\hspace{0pt}\bf Observation}
\newtheorem{theorem}{\hspace{0pt}\bf Theorem}
%\newtheorem{corollary}{\hspace{0pt}\bf Corollary}
\newtheorem{fact}{\hspace{0pt}\bf Fact}

%\newcounter{remark} \setcounter{remark}{1}
%\newenvironment{remark}{\vspace{-0.1cm} \noindent {\bf Remark
%\theremark :} \addtocounter{remark}{-1} \refstepcounter{remark}} {
%\vspace{-0.1cm} \normalfont \addtocounter{remark}{1} }

\newtheorem{remark}{\hspace{-11pt}\bf Remark}
%\newtheorem{remark}{\noindent \bf Remark}

\newtheorem{test}{\hspace{0pt}\it Test Case}
\newtheorem{definition}{\hspace{0pt}\bf Definition}
\newtheorem{example}{\hspace{0pt}\bf Example}

\def\proof      {\noindent\hspace{0pt}{\bf{Proof: }}}
\def\myQED      {\hfill$\square$\vspace{0.3cm}}








%%%%%%%%%%%%%%%%%%%%%%%%%%%%%%%%%bar version
%capital alphabet
\def\bbarA{{\ensuremath{\bar A}}}
\def\bbarB{{\ensuremath{\bar B}}}
\def\bbarC{{\ensuremath{\bar C}}}
\def\bbarD{{\ensuremath{\bar D}}}
\def\bbarE{{\ensuremath{\bar E}}}
\def\bbarF{{\ensuremath{\bar F}}}
\def\bbarG{{\ensuremath{\bar G}}}
\def\bbarH{{\ensuremath{\bar H}}}
\def\bbarI{{\ensuremath{\bar I}}}
\def\bbarJ{{\ensuremath{\bar J}}}
\def\bbarK{{\ensuremath{\bar K}}}
\def\bbarL{{\ensuremath{\bar L}}}
\def\bbarM{{\ensuremath{\bar M}}}
\def\bbarN{{\ensuremath{\bar N}}}
\def\bbarO{{\ensuremath{\bar O}}}
\def\bbarP{{\ensuremath{\bar P}}}
\def\bbarQ{{\ensuremath{\bar Q}}}
\def\bbarR{{\ensuremath{\bar R}}}
\def\bbarW{{\ensuremath{\bar W}}}
\def\bbarU{{\ensuremath{\bar U}}}
\def\bbarV{{\ensuremath{\bar V}}}
\def\bbarS{{\ensuremath{\bar S}}}
\def\bbarT{{\ensuremath{\bar T}}}
\def\bbarX{{\ensuremath{\bar X}}}
\def\bbarY{{\ensuremath{\bar Y}}}
\def\bbarZ{{\ensuremath{\bar Z}}}
%lower case alphabet
\def\bbara{{\ensuremath{\bar a}}}
\def\bbarb{{\ensuremath{\bar b}}}
\def\bbarc{{\ensuremath{\bar c}}}
\def\bbard{{\ensuremath{\bar d}}}
\def\bbare{{\ensuremath{\bar e}}}
\def\bbarf{{\ensuremath{\bar f}}}
\def\bbarg{{\ensuremath{\bar g}}}
\def\bbarh{{\ensuremath{\bar h}}}
\def\bbari{{\ensuremath{\bar i}}}
\def\bbarj{{\ensuremath{\bar j}}}
\def\bbark{{\ensuremath{\bar k}}}
\def\bbarl{{\ensuremath{\bar l}}}
\def\bbarm{{\ensuremath{\bar m}}}
\def\bbarn{{\ensuremath{\bar n}}}
\def\bbaro{{\ensuremath{\bar o}}}
\def\bbarp{{\ensuremath{\bar p}}}
\def\bbarq{{\ensuremath{\bar q}}}
\def\bbarr{{\ensuremath{\bar r}}}
\def\bbarw{{\ensuremath{\bar w}}}
\def\bbaru{{\ensuremath{\bar u}}}
\def\bbarv{{\ensuremath{\bar v}}}
\def\bbars{{\ensuremath{\bar s}}}
\def\bbart{{\ensuremath{\bar t}}}
\def\bbarx{{\ensuremath{\bar x}}}
\def\bbary{{\ensuremath{\bar y}}}
\def\bbarz{{\ensuremath{\bar z}}}
%%%%%%%%%%%%%%%%%%%%%%%%%%%%%%%%%%end of bar version
%
%%%%%%%%%%%%%%%%%%%%%%%%%%%%%%%%%%%%%caligraph version
\def\ccalA{{\ensuremath{\mathcal A}}}
\def\ccalB{{\ensuremath{\mathcal B}}}
\def\ccalC{{\ensuremath{\mathcal C}}}
\def\ccalD{{\ensuremath{\mathcal D}}}
\def\ccalE{{\ensuremath{\mathcal E}}}
\def\ccalF{{\ensuremath{\mathcal F}}}
\def\ccalG{{\ensuremath{\mathcal G}}}
\def\ccalH{{\ensuremath{\mathcal H}}}
\def\ccalI{{\ensuremath{\mathcal I}}}
\def\ccalJ{{\ensuremath{\mathcal J}}}
\def\ccalK{{\ensuremath{\mathcal K}}}
\def\ccalL{{\ensuremath{\mathcal L}}}
\def\ccalM{{\ensuremath{\mathcal M}}}
\def\ccalN{{\ensuremath{\mathcal N}}}
\def\ccalO{{\ensuremath{\mathcal O}}}
\def\ccalP{{\ensuremath{\mathcal P}}}
\def\ccalQ{{\ensuremath{\mathcal Q}}}
\def\ccalR{{\ensuremath{\mathcal R}}}
\def\ccalW{{\ensuremath{\mathcal W}}}
\def\ccalU{{\ensuremath{\mathcal U}}}
\def\ccalV{{\ensuremath{\mathcal V}}}
\def\ccalS{{\ensuremath{\mathcal S}}}
\def\ccalT{{\ensuremath{\mathcal T}}}
\def\ccalX{{\ensuremath{\mathcal X}}}
\def\ccalY{{\ensuremath{\mathcal Y}}}
\def\ccalZ{{\ensuremath{\mathcal Z}}}
%lower case alphabet
\def\ccala{{\ensuremath{\mathcal a}}}
\def\ccalb{{\ensuremath{\mathcal b}}}
\def\ccalc{{\ensuremath{\mathcal c}}}
\def\ccald{{\ensuremath{\mathcal d}}}
\def\ccale{{\ensuremath{\mathcal e}}}
\def\ccalf{{\ensuremath{\mathcal f}}}
\def\ccalg{{\ensuremath{\mathcal g}}}
\def\ccalh{{\ensuremath{\mathcal h}}}
\def\ccali{{\ensuremath{\mathcal i}}}
\def\ccalj{{\ensuremath{\mathcal j}}}
\def\ccalk{{\ensuremath{\mathcal k}}}
\def\ccall{{\ensuremath{\mathcal l}}}
\def\ccalm{{\ensuremath{\mathcal m}}}
\def\ccaln{{\ensuremath{\mathcal n}}}
\def\ccalo{{\ensuremath{\mathcal o}}}
\def\ccalp{{\ensuremath{\mathcal p}}}
\def\ccalq{{\ensuremath{\mathcal q}}}
\def\ccalr{{\ensuremath{\mathcal r}}}
\def\ccalw{{\ensuremath{\mathcal w}}}
\def\ccalu{{\ensuremath{\mathcal u}}}
\def\ccalv{{\ensuremath{\mathcal v}}}
\def\ccals{{\ensuremath{\mathcal s}}}
\def\ccalt{{\ensuremath{\mathcal t}}}
\def\ccalx{{\ensuremath{\mathcal x}}}
\def\ccaly{{\ensuremath{\mathcal y}}}
\def\ccalz{{\ensuremath{\mathcal z}}}
\def\ccal0{{\ensuremath{\mathcal 0}}}
%%%%%%%%%%%%%%%%%%%%%%%%%%%%%%%%%%%%%%%%%end of caligraph version
%
%
%%%%%%%%%%%%%%%%%%%%%%%%%%%%%%%%%%%%%%%%%%%hat version
%capital alphabet
\def\hhatA{{\ensuremath{\hat A}}}
\def\hhatB{{\ensuremath{\hat B}}}
\def\hhatC{{\ensuremath{\hat C}}}
\def\hhatD{{\ensuremath{\hat D}}}
\def\hhatE{{\ensuremath{\hat E}}}
\def\hhatF{{\ensuremath{\hat F}}}
\def\hhatG{{\ensuremath{\hat G}}}
\def\hhatH{{\ensuremath{\hat H}}}
\def\hhatI{{\ensuremath{\hat I}}}
\def\hhatJ{{\ensuremath{\hat J}}}
\def\hhatK{{\ensuremath{\hat K}}}
\def\hhatL{{\ensuremath{\hat L}}}
\def\hhatM{{\ensuremath{\hat M}}}
\def\hhatN{{\ensuremath{\hat N}}}
\def\hhatO{{\ensuremath{\hat O}}}
\def\hhatP{{\ensuremath{\hat P}}}
\def\hhatQ{{\ensuremath{\hat Q}}}
\def\hhatR{{\ensuremath{\hat R}}}
\def\hhatW{{\ensuremath{\hat W}}}
\def\hhatU{{\ensuremath{\hat U}}}
\def\hhatV{{\ensuremath{\hat V}}}
\def\hhatS{{\ensuremath{\hat S}}}
\def\hhatT{{\ensuremath{\hat T}}}
\def\hhatX{{\ensuremath{\hat X}}}
\def\hhatY{{\ensuremath{\hat Y}}}
\def\hhatZ{{\ensuremath{\hat Z}}}
%lower case alphabet
\def\hhata{{\ensuremath{\hat a}}}
\def\hhatb{{\ensuremath{\hat b}}}
\def\hhatc{{\ensuremath{\hat c}}}
\def\hhatd{{\ensuremath{\hat d}}}
\def\hhate{{\ensuremath{\hat e}}}
\def\hhatf{{\ensuremath{\hat f}}}
\def\hhatg{{\ensuremath{\hat g}}}
\def\hhath{{\ensuremath{\hat h}}}
\def\hhati{{\ensuremath{\hat i}}}
\def\hhatj{{\ensuremath{\hat j}}}
\def\hhatk{{\ensuremath{\hat k}}}
\def\hhatl{{\ensuremath{\hat l}}}
\def\hhatm{{\ensuremath{\hat m}}}
\def\hhatn{{\ensuremath{\hat n}}}
\def\hhato{{\ensuremath{\hat o}}}
\def\hhatp{{\ensuremath{\hat p}}}
\def\hhatq{{\ensuremath{\hat q}}}
\def\hhatr{{\ensuremath{\hat r}}}
\def\hhatw{{\ensuremath{\hat w}}}
\def\hhatu{{\ensuremath{\hat u}}}
\def\hhatv{{\ensuremath{\hat v}}}
\def\hhats{{\ensuremath{\hat s}}}
\def\hhatt{{\ensuremath{\hat t}}}
\def\hhatx{{\ensuremath{\hat x}}}
\def\hhaty{{\ensuremath{\hat y}}}
\def\hhatz{{\ensuremath{\hat z}}}
%%%%%%%%%%%%%%%%%%%%%%%%%%%%%%%%%%end of hat version
%
%
%%%%%%%%%%%%%%%%%%%%%%%%%%%%%%%%%%tilde version
%capital alphabet
\def\tdA{{\ensuremath{\tilde A}}}
\def\tdB{{\ensuremath{\tilde B}}}
\def\tdC{{\ensuremath{\tilde C}}}
\def\tdD{{\ensuremath{\tilde D}}}
\def\tdE{{\ensuremath{\tilde E}}}
\def\tdF{{\ensuremath{\tilde F}}}
\def\tdG{{\ensuremath{\tilde G}}}
\def\tdH{{\ensuremath{\tilde H}}}
\def\tdI{{\ensuremath{\tilde I}}}
\def\tdJ{{\ensuremath{\tilde J}}}
\def\tdK{{\ensuremath{\tilde K}}}
\def\tdL{{\ensuremath{\tilde L}}}
\def\tdM{{\ensuremath{\tilde M}}}
\def\tdN{{\ensuremath{\tilde N}}}
\def\tdO{{\ensuremath{\tilde O}}}
\def\tdP{{\ensuremath{\tilde P}}}
\def\tdQ{{\ensuremath{\tilde Q}}}
\def\tdR{{\ensuremath{\tilde R}}}
\def\tdW{{\ensuremath{\tilde W}}}
\def\tdU{{\ensuremath{\tilde U}}}
\def\tdV{{\ensuremath{\tilde V}}}
\def\tdS{{\ensuremath{\tilde S}}}
\def\tdT{{\ensuremath{\tilde T}}}
\def\tdX{{\ensuremath{\tilde X}}}
\def\tdY{{\ensuremath{\tilde Y}}}
\def\tdZ{{\ensuremath{\tilde Z}}}
%lower case alphabet
\def\tda{{\ensuremath{\tilde a}}}
\def\tdb{{\ensuremath{\tilde b}}}
\def\tdc{{\ensuremath{\tilde c}}}
\def\tdd{{\ensuremath{\tilde d}}}
\def\tde{{\ensuremath{\tilde e}}}
\def\tdf{{\ensuremath{\tilde f}}}
\def\tdg{{\ensuremath{\tilde g}}}
\def\tdh{{\ensuremath{\tilde h}}}
\def\tdi{{\ensuremath{\tilde i}}}
\def\tdj{{\ensuremath{\tilde j}}}
\def\tdk{{\ensuremath{\tilde k}}}
\def\tdl{{\ensuremath{\tilde l}}}
\def\tdm{{\ensuremath{\tilde m}}}
\def\tdn{{\ensuremath{\tilde n}}}
\def\tdo{{\ensuremath{\tilde o}}}
\def\tdp{{\ensuremath{\tilde p}}}
\def\tdq{{\ensuremath{\tilde q}}}
\def\tdr{{\ensuremath{\tilde r}}}
\def\tdw{{\ensuremath{\tilde w}}}
\def\tdu{{\ensuremath{\tilde u}}}
\def\tdv{{\ensuremath{\tilde r}}}
\def\tds{{\ensuremath{\tilde s}}}
\def\tdt{{\ensuremath{\tilde t}}}
\def\tdx{{\ensuremath{\tilde x}}}
\def\tdy{{\ensuremath{\tilde y}}}
\def\tdz{{\ensuremath{\tilde z}}}
%%%%%%%%%%%%%%%%%%%%%%%%%%%%%%%%%%%%end of tilde version
%
%%%%%%%%%%%%%%%%%%%%%%%%%%%%%%%%%%%%%check version
%lower case alphabet
\def\chka{{\ensuremath{\check a}}}
\def\chkb{{\ensuremath{\check b}}}
\def\chkc{{\ensuremath{\check c}}}
\def\chkd{{\ensuremath{\check d}}}
\def\chke{{\ensuremath{\check e}}}
\def\chkf{{\ensuremath{\check f}}}
\def\chkg{{\ensuremath{\check g}}}
\def\chkh{{\ensuremath{\check h}}}
\def\chki{{\ensuremath{\check i}}}
\def\chkj{{\ensuremath{\check j}}}
\def\chkk{{\ensuremath{\check k}}}
\def\chkl{{\ensuremath{\check l}}}
\def\chkm{{\ensuremath{\check m}}}
\def\chkn{{\ensuremath{\check n}}}
\def\chko{{\ensuremath{\check o}}}
\def\chkp{{\ensuremath{\check p}}}
\def\chkq{{\ensuremath{\check q}}}
\def\chkr{{\ensuremath{\check r}}}
\def\chkw{{\ensuremath{\check w}}}
\def\chku{{\ensuremath{\check u}}}
\def\chkv{{\ensuremath{\check v}}}
\def\chks{{\ensuremath{\check s}}}
\def\chkt{{\ensuremath{\check t}}}
\def\chkx{{\ensuremath{\check x}}}
\def\chky{{\ensuremath{\check y}}}
\def\chkz{{\ensuremath{\check z}}}
%%%%%%%%%%%%%%%%%%%%%%%%%%%%%%%%%%end of check version
%
%
%%%%%%%%%%%%%%%%%%%%%%%%%%%%%%%%%%%%Bold version
% upper case bold
\def\bbA{{\ensuremath{\mathbf A}}}
\def\bbB{{\ensuremath{\mathbf B}}}
\def\bbC{{\ensuremath{\mathbf C}}}
\def\bbD{{\ensuremath{\mathbf D}}}
\def\bbE{{\ensuremath{\mathbf E}}}
\def\bbF{{\ensuremath{\mathbf F}}}
\def\bbG{{\ensuremath{\mathbf G}}}
\def\bbH{{\ensuremath{\mathbf H}}}
\def\bbI{{\ensuremath{\mathbf I}}}
\def\bbJ{{\ensuremath{\mathbf J}}}
\def\bbK{{\ensuremath{\mathbf K}}}
\def\bbL{{\ensuremath{\mathbf L}}}
\def\bbM{{\ensuremath{\mathbf M}}}
\def\bbN{{\ensuremath{\mathbf N}}}
\def\bbO{{\ensuremath{\mathbf O}}}
\def\bbP{{\ensuremath{\mathbf P}}}
\def\bbQ{{\ensuremath{\mathbf Q}}}
\def\bbR{{\ensuremath{\mathbf R}}}
\def\bbW{{\ensuremath{\mathbf W}}}
\def\bbU{{\ensuremath{\mathbf U}}}
\def\bbV{{\ensuremath{\mathbf V}}}
\def\bbS{{\ensuremath{\mathbf S}}}
\def\bbT{{\ensuremath{\mathbf T}}}
\def\bbX{{\ensuremath{\mathbf X}}}
\def\bbY{{\ensuremath{\mathbf Y}}}
\def\bbZ{{\ensuremath{\mathbf Z}}}
%lower case bold
\def\bba{{\ensuremath{\mathbf a}}}
\def\bbb{{\ensuremath{\mathbf b}}}
\def\bbc{{\ensuremath{\mathbf c}}}
\def\bbd{{\ensuremath{\mathbf d}}}
\def\bbe{{\ensuremath{\mathbf e}}}
\def\bbf{{\ensuremath{\mathbf f}}}
\def\bbg{{\ensuremath{\mathbf g}}}
\def\bbh{{\ensuremath{\mathbf h}}}
\def\bbi{{\ensuremath{\mathbf i}}}
\def\bbj{{\ensuremath{\mathbf j}}}
\def\bbk{{\ensuremath{\mathbf k}}}
\def\bbl{{\ensuremath{\mathbf l}}}
\def\bbm{{\ensuremath{\mathbf m}}}
\def\bbn{{\ensuremath{\mathbf n}}}
\def\bbo{{\ensuremath{\mathbf o}}}
\def\bbp{{\ensuremath{\mathbf p}}}
\def\bbq{{\ensuremath{\mathbf q}}}
\def\bbr{{\ensuremath{\mathbf r}}}
\def\bbw{{\ensuremath{\mathbf w}}}
\def\bbu{{\ensuremath{\mathbf u}}}
\def\bbv{{\ensuremath{\mathbf v}}}
\def\bbs{{\ensuremath{\mathbf s}}}
\def\bbt{{\ensuremath{\mathbf t}}}
\def\bbx{{\ensuremath{\mathbf x}}}
\def\bby{{\ensuremath{\mathbf y}}}
\def\bbz{{\ensuremath{\mathbf z}}}
\def\bb0{{\ensuremath{\mathbf 0}}}
%

%%%%%%%%%%%%%%%%%%%%%%%%%%%%%%%%%%%%%%%%%%%%%%bar bold version
%upper case
%
\def\bbarbbA{{\bar{\ensuremath{\mathbf A}} }}
\def\bbarbbB{{\bar{\ensuremath{\mathbf B}} }}
\def\bbarbbC{{\bar{\ensuremath{\mathbf C}} }}
\def\bbarbbD{{\bar{\ensuremath{\mathbf D}} }}
\def\bbarbbE{{\bar{\ensuremath{\mathbf E}} }}
\def\bbarbbF{{\bar{\ensuremath{\mathbf F}} }}
\def\bbarbbG{{\bar{\ensuremath{\mathbf G}} }}
\def\bbarbbH{{\bar{\ensuremath{\mathbf H}} }}
\def\bbarbbI{{\bar{\ensuremath{\mathbf I}} }}
\def\bbarbbJ{{\bar{\ensuremath{\mathbf J}} }}
\def\bbarbbK{{\bar{\ensuremath{\mathbf K}} }}
\def\bbarbbL{{\bar{\ensuremath{\mathbf L}} }}
\def\bbarbbM{{\bar{\ensuremath{\mathbf M}} }}
\def\bbarbbN{{\bar{\ensuremath{\mathbf N}} }}
\def\bbarbbO{{\bar{\ensuremath{\mathbf O}} }}
\def\bbarbbP{{\bar{\ensuremath{\mathbf P}} }}
\def\bbarbbQ{{\bar{\ensuremath{\mathbf Q}} }}
\def\bbarbbR{{\bar{\ensuremath{\mathbf R}} }}
\def\bbarbbS{{\bar{\ensuremath{\mathbf S}} }}
\def\bbarbbT{{\bar{\ensuremath{\mathbf T}} }}
\def\bbarbbU{{\bar{\ensuremath{\mathbf U}} }}
\def\bbarbbV{{\bar{\ensuremath{\mathbf V}} }}
\def\bbarbbW{{\bar{\ensuremath{\mathbf W}} }}
\def\bbarbbX{{\bar{\ensuremath{\mathbf X}} }}
\def\bbarbbY{{\bar{\ensuremath{\mathbf Y}} }}
\def\bbarbbZ{{\bar{\ensuremath{\mathbf Z}} }}
%
%lower case
%
\def\bbarbba{{\bar{\ensuremath{\mathbf a}} }}
\def\bbarbbb{{\bar{\ensuremath{\mathbf b}} }}
\def\bbarbbc{{\bar{\ensuremath{\mathbf c}} }}
\def\bbarbbd{{\bar{\ensuremath{\mathbf d}} }}
\def\bbarbbe{{\bar{\ensuremath{\mathbf e}} }}
\def\bbarbbf{{\bar{\ensuremath{\mathbf f}} }}
\def\bbarbbg{{\bar{\ensuremath{\mathbf g}} }}
\def\bbarbbh{{\bar{\ensuremath{\mathbf h}} }}
\def\bbarbbi{{\bar{\ensuremath{\mathbf i}} }}
\def\bbarbbj{{\bar{\ensuremath{\mathbf j}} }}
\def\bbarbbk{{\bar{\ensuremath{\mathbf k}} }}
\def\bbarbbl{{\bar{\ensuremath{\mathbf l}} }}
\def\bbarbbm{{\bar{\ensuremath{\mathbf m}} }}
\def\bbarbbn{{\bar{\ensuremath{\mathbf n}} }}
\def\bbarbbo{{\bar{\ensuremath{\mathbf o}} }}
\def\bbarbbp{{\bar{\ensuremath{\mathbf p}} }}
\def\bbarbbq{{\bar{\ensuremath{\mathbf q}} }}
\def\bbarbbr{{\bar{\ensuremath{\mathbf r}} }}
\def\bbarbbs{{\bar{\ensuremath{\mathbf s}} }}
\def\bbarbbt{{\bar{\ensuremath{\mathbf t}} }}
\def\bbarbbu{{\bar{\ensuremath{\mathbf u}} }}
\def\bbarbbv{{\bar{\ensuremath{\mathbf v}} }}
\def\bbarbbw{{\bar{\ensuremath{\mathbf w}} }}
\def\bbarbbx{{\bar{\ensuremath{\mathbf x}} }}
\def\bbarbby{{\bar{\ensuremath{\mathbf y}} }}
\def\bbarbbz{{\bar{\ensuremath{\mathbf z}} }}
%%%%%%%%%%%%%%%%%%%%%%%%%%%%%%%%%%%%%%%%%%%%%%%end of bar bold bersion
%
%
%%%%%%%%%%%%%%%%%%%%%%%%%%%%%%%%%%%%%%%%%%%%%%hat bold version
%upper case
%
\def\hhatbbA{{\hat{\ensuremath{\mathbf A}} }}
\def\hhatbbB{{\hat{\ensuremath{\mathbf B}} }}
\def\hhatbbC{{\hat{\ensuremath{\mathbf C}} }}
\def\hhatbbD{{\hat{\ensuremath{\mathbf D}} }}
\def\hhatbbE{{\hat{\ensuremath{\mathbf E}} }}
\def\hhatbbF{{\hat{\ensuremath{\mathbf F}} }}
\def\hhatbbG{{\hat{\ensuremath{\mathbf G}} }}
\def\hhatbbH{{\hat{\ensuremath{\mathbf H}} }}
\def\hhatbbI{{\hat{\ensuremath{\mathbf I}} }}
\def\hhatbbJ{{\hat{\ensuremath{\mathbf J}} }}
\def\hhatbbK{{\hat{\ensuremath{\mathbf K}} }}
\def\hhatbbL{{\hat{\ensuremath{\mathbf L}} }}
\def\hhatbbM{{\hat{\ensuremath{\mathbf M}} }}
\def\hhatbbN{{\hat{\ensuremath{\mathbf N}} }}
\def\hhatbbO{{\hat{\ensuremath{\mathbf O}} }}
\def\hhatbbP{{\hat{\ensuremath{\mathbf P}} }}
\def\hhatbbQ{{\hat{\ensuremath{\mathbf Q}} }}
\def\hhatbbR{{\hat{\ensuremath{\mathbf R}} }}
\def\hhatbbS{{\hat{\ensuremath{\mathbf S}} }}
\def\hhatbbT{{\hat{\ensuremath{\mathbf T}} }}
\def\hhatbbU{{\hat{\ensuremath{\mathbf U}} }}
\def\hhatbbV{{\hat{\ensuremath{\mathbf V}} }}
\def\hhatbbW{{\hat{\ensuremath{\mathbf W}} }}
\def\hhatbbX{{\hat{\ensuremath{\mathbf X}} }}
\def\hhatbbY{{\hat{\ensuremath{\mathbf Y}} }}
\def\hhatbbZ{{\hat{\ensuremath{\mathbf Z}} }}
%
%lower case
%
\def\hhatbba{{\hat{\ensuremath{\mathbf a}} }}
\def\hhatbbb{{\hat{\ensuremath{\mathbf b}} }}
\def\hhatbbc{{\hat{\ensuremath{\mathbf c}} }}
\def\hhatbbd{{\hat{\ensuremath{\mathbf d}} }}
\def\hhatbbe{{\hat{\ensuremath{\mathbf e}} }}
\def\hhatbbf{{\hat{\ensuremath{\mathbf f}} }}
\def\hhatbbg{{\hat{\ensuremath{\mathbf g}} }}
\def\hhatbbh{{\hat{\ensuremath{\mathbf h}} }}
\def\hhatbbi{{\hat{\ensuremath{\mathbf i}} }}
\def\hhatbbj{{\hat{\ensuremath{\mathbf j}} }}
\def\hhatbbk{{\hat{\ensuremath{\mathbf k}} }}
\def\hhatbbl{{\hat{\ensuremath{\mathbf l}} }}
\def\hhatbbm{{\hat{\ensuremath{\mathbf m}} }}
\def\hhatbbn{{\hat{\ensuremath{\mathbf n}} }}
\def\hhatbbo{{\hat{\ensuremath{\mathbf o}} }}
\def\hhatbbp{{\hat{\ensuremath{\mathbf p}} }}
\def\hhatbbq{{\hat{\ensuremath{\mathbf q}} }}
\def\hhatbbr{{\hat{\ensuremath{\mathbf r}} }}
\def\hhatbbs{{\hat{\ensuremath{\mathbf s}} }}
\def\hhatbbt{{\hat{\ensuremath{\mathbf t}} }}
\def\hhatbbu{{\hat{\ensuremath{\mathbf u}} }}
\def\hhatbbv{{\hat{\ensuremath{\mathbf v}} }}
\def\hhatbbw{{\hat{\ensuremath{\mathbf w}} }}
\def\hhatbbx{{\hat{\ensuremath{\mathbf x}} }}
\def\hhatbby{{\hat{\ensuremath{\mathbf y}} }}
\def\hhatbbz{{\hat{\ensuremath{\mathbf z}} }}
%%%%%%%%%%%%%%%%%%%%%%%%%%%%%%%%%%%%%%%%%%%%%%%end of hat bold  bersion
%
%
%%%%%%%%%%%%%%%%%%%%%%%%%%%%%%%%%%%%%%%%%%%%%%tilde bold version
%upper case
%
\def\tdbbA{{\tilde{\ensuremath{\mathbf A}} }}
\def\tdbbB{{\tilde{\ensuremath{\mathbf B}} }}
\def\tdbbC{{\tilde{\ensuremath{\mathbf C}} }}
\def\tdbbD{{\tilde{\ensuremath{\mathbf D}} }}
\def\tdbbE{{\tilde{\ensuremath{\mathbf E}} }}
\def\tdbbF{{\tilde{\ensuremath{\mathbf F}} }}
\def\tdbbG{{\tilde{\ensuremath{\mathbf G}} }}
\def\tdbbH{{\tilde{\ensuremath{\mathbf H}} }}
\def\tdbbI{{\tilde{\ensuremath{\mathbf I}} }}
\def\tdbbJ{{\tilde{\ensuremath{\mathbf J}} }}
\def\tdbbK{{\tilde{\ensuremath{\mathbf K}} }}
\def\tdbbL{{\tilde{\ensuremath{\mathbf L}} }}
\def\tdbbM{{\tilde{\ensuremath{\mathbf M}} }}
\def\tdbbN{{\tilde{\ensuremath{\mathbf N}} }}
\def\tdbbO{{\tilde{\ensuremath{\mathbf O}} }}
\def\tdbbP{{\tilde{\ensuremath{\mathbf P}} }}
\def\tdbbQ{{\tilde{\ensuremath{\mathbf Q}} }}
\def\tdbbR{{\tilde{\ensuremath{\mathbf R}} }}
\def\tdbbS{{\tilde{\ensuremath{\mathbf S}} }}
\def\tdbbT{{\tilde{\ensuremath{\mathbf T}} }}
\def\tdbbU{{\tilde{\ensuremath{\mathbf U}} }}
\def\tdbbV{{\tilde{\ensuremath{\mathbf V}} }}
\def\tdbbW{{\tilde{\ensuremath{\mathbf W}} }}
\def\tdbbX{{\tilde{\ensuremath{\mathbf X}} }}
\def\tdbbY{{\tilde{\ensuremath{\mathbf Y}} }}
\def\tdbbZ{{\tilde{\ensuremath{\mathbf Z}} }}
%
%lower case
%
\def\tdbba{{\tilde{\ensuremath{\mathbf a}} }}
\def\tdbbb{{\tilde{\ensuremath{\mathbf b}} }}
\def\tdbbc{{\tilde{\ensuremath{\mathbf c}} }}
\def\tdbbd{{\tilde{\ensuremath{\mathbf d}} }}
\def\tdbbe{{\tilde{\ensuremath{\mathbf e}} }}
\def\tdbbf{{\tilde{\ensuremath{\mathbf f}} }}
\def\tdbbg{{\tilde{\ensuremath{\mathbf g}} }}
\def\tdbbh{{\tilde{\ensuremath{\mathbf h}} }}
\def\tdbbi{{\tilde{\ensuremath{\mathbf i}} }}
\def\tdbbj{{\tilde{\ensuremath{\mathbf j}} }}
\def\tdbbk{{\tilde{\ensuremath{\mathbf k}} }}
\def\tdbbl{{\tilde{\ensuremath{\mathbf l}} }}
\def\tdbbm{{\tilde{\ensuremath{\mathbf m}} }}
\def\tdbbn{{\tilde{\ensuremath{\mathbf n}} }}
\def\tdbbo{{\tilde{\ensuremath{\mathbf o}} }}
\def\tdbbp{{\tilde{\ensuremath{\mathbf p}} }}
\def\tdbbq{{\tilde{\ensuremath{\mathbf q}} }}
\def\tdbbr{{\tilde{\ensuremath{\mathbf r}} }}
\def\tdbbs{{\tilde{\ensuremath{\mathbf s}} }}
\def\tdbbt{{\tilde{\ensuremath{\mathbf t}} }}
\def\tdbbu{{\tilde{\ensuremath{\mathbf u}} }}
\def\tdbbv{{\tilde{\ensuremath{\mathbf v}} }}
\def\tdbbw{{\tilde{\ensuremath{\mathbf w}} }}
\def\tdbbx{{\tilde{\ensuremath{\mathbf x}} }}
\def\tdbby{{\tilde{\ensuremath{\mathbf y}} }}
\def\tdbbz{{\tilde{\ensuremath{\mathbf z}} }}
%%%%%%%%%%%%%%%%%%%%%%%%%%%%%%%%%%%%%%%%%%%%%%%end of tilde bold  bersion
%
%%%%%%%%%%%%%%%%%%%%%%%%%%%%%%%%%%%%%%%%%%%%%%%bold caligraph
%
\def\bbcalA{\mbox{\boldmath $\mathcal{A}$}}
\def\bbcalB{\mbox{\boldmath $\mathcal{B}$}}
\def\bbcalC{\mbox{\boldmath $\mathcal{C}$}}
\def\bbcalD{\mbox{\boldmath $\mathcal{D}$}}
\def\bbcalE{\mbox{\boldmath $\mathcal{E}$}}
\def\bbcalF{\mbox{\boldmath $\mathcal{F}$}}
\def\bbcalG{\mbox{\boldmath $\mathcal{G}$}}
\def\bbcalH{\mbox{\boldmath $\mathcal{H}$}}
\def\bbcalI{\mbox{\boldmath $\mathcal{I}$}}
\def\bbcalJ{\mbox{\boldmath $\mathcal{J}$}}
\def\bbcalK{\mbox{\boldmath $\mathcal{K}$}}
\def\bbcalL{\mbox{\boldmath $\mathcal{L}$}}
\def\bbcalM{\mbox{\boldmath $\mathcal{M}$}}
\def\bbcalN{\mbox{\boldmath $\mathcal{N}$}}
\def\bbcalO{\mbox{\boldmath $\mathcal{O}$}}
\def\bbcalP{\mbox{\boldmath $\mathcal{P}$}}
\def\bbcalQ{\mbox{\boldmath $\mathcal{Q}$}}
\def\bbcalR{\mbox{\boldmath $\mathcal{R}$}}
\def\bbcalW{\mbox{\boldmath $\mathcal{W}$}}
\def\bbcalU{\mbox{\boldmath $\mathcal{U}$}}
\def\bbcalV{\mbox{\boldmath $\mathcal{V}$}}
\def\bbcalS{\mbox{\boldmath $\mathcal{S}$}}
\def\bbcalT{\mbox{\boldmath $\mathcal{T}$}}
\def\bbcalX{\mbox{\boldmath $\mathcal{X}$}}
\def\bbcalY{\mbox{\boldmath $\mathcal{Y}$}}
\def\bbcalZ{\mbox{\boldmath $\mathcal{Z}$}}
%
%%%%%%%%%%%%%%%%%%%%%%%%%%%%%%%%%%%%%%%%%%%%%%%%%%end of caligraph
%
%
%
%
%%%%%%%%%%%%%%%%%%%%%%%%%%%%%%%%%%%%%%%%%%%%%%%tilde Greek
%
\def\tdupsilon{\tilde\upsilon}
\def\tdalpha{\tilde\alpha}
\def\tdbeta{\tilde\beta}
\def\tdgamma{\tilde\gamma}
\def\tddelta{\tilde\delta}
\def\tdepsilon{\tilde\epsilon}
\def\tdvarepsilon{\tilde\varepsilon}
\def\tdzeta{\tilde\zeta}
\def\tdeta{\tilde\eta}
\def\tdtheta{\tilde\theta}
\def\tdvartheta{\tilde\vartheta}

\def\tdiota{\tilde\iota}
\def\tdkappa{\tilde\kappa}
\def\tdlambda{\tilde\lambda}
\def\tdmu{\tilde\mu}
\def\tdnu{\tilde\nu}
\def\tdxi{\tilde\xi}
\def\tdpi{\tilde\pi}
\def\tdrho{\tilde\rho}
\def\tdvarrho{\tilde\varrho}
\def\tdsigma{\tilde\sigma}
\def\tdvarsigma{\tilde\varsigma}
\def\tdtau{\tilde\tau}
\def\tdupsilon{\tilde\upsilon}
\def\tdphi{\tilde\phi}
\def\tdvarphi{\tilde\varphi}
\def\tdchi{\tilde\chi}
\def\tdpsi{\tilde\psi}
\def\tdomega{\tilde\omega}

\def\tdGamma{\tilde\Gamma}
\def\tdDelta{\tilde\Delta}
\def\tdTheta{\tilde\Theta}
\def\tdLambda{\tilde\Lambda}
\def\tdXi{\tilde\Xi}
\def\tdPi{\tilde\Pi}
\def\tdSigma{\tilde\Sigma}
\def\tdUpsilon{\tilde\Upsilon}
\def\tdPhi{\tilde\Phi}
\def\tdPsi{\tilde\Psi}
%%%%%%%%%%%%%%%%%%%%%%%%%%%%%%%%%%%%%%%%%%%end of title  Greek
%
%%%%%%%%%%%%%%%%%%%%%%%%%%%%%%%%%%%%%%%%%%%%%%%bar Greek
%
\def\bbarupsilon{\bar\upsilon}
\def\bbaralpha{\bar\alpha}
\def\bbarbeta{\bar\beta}
\def\bbargamma{\bar\gamma}
\def\bbardelta{\bar\delta}
\def\bbarepsilon{\bar\epsilon}
\def\bbarvarepsilon{\bar\varepsilon}
\def\bbarzeta{\bar\zeta}
\def\bbareta{\bar\eta}
\def\bbartheta{\bar\theta}
\def\bbarvartheta{\bar\vartheta}

\def\bbariota{\bar\iota}
\def\bbarkappa{\bar\kappa}
\def\bbarlambda{\bar\lambda}
\def\bbarmu{\bar\mu}
\def\bbarnu{\bar\nu}
\def\bbarxi{\bar\xi}
\def\bbarpi{\bar\pi}
\def\bbarrho{\bar\rho}
\def\bbarvarrho{\bar\varrho}
\def\bbarvarsigma{\bar\varsigma}
\def\bbarphi{\bar\phi}
\def\bbarvarphi{\bar\varphi}
\def\bbarchi{\bar\chi}
\def\bbarpsi{\bar\psi}
\def\bbaromega{\bar\omega}

\def\bbarGamma{\bar\Gamma}
\def\bbarDelta{\bar\Delta}
\def\bbarTheta{\bar\Theta}
\def\bbarLambda{\bar\Lambda}
\def\bbarXi{\bar\Xi}
\def\bbarPi{\bar\Pi}
\def\bbarSigma{\bar\Sigma}
\def\bbarUpsilon{\bar\Upsilon}
\def\bbarPhi{\bar\Phi}
\def\bbarPsi{\bar\Psi}
%%%%%%%%%%%%%%%%%%%%%%%%%%%%%%%%%%%%%%%%%%%end of bar  Greek
%
%
%
%%%%%%%%%%%%%%%%%%%%%%%%%%%%%%%%%%%%%%%%%%%%%%%begion of check Greek
%
\def\chkupsilon{\check\upsilon}
\def\chkalpha{\check\alpha}
\def\chkbeta{\check\beta}
\def\chkgamma{\check\gamma}
\def\chkdelta{\check\delta}
\def\chkepsilon{\check\epsilon}
\def\chkvarepsilon{\check\varepsilon}
\def\chkzeta{\check\zeta}
\def\chketa{\check\eta}
\def\chktheta{\check\theta}
\def\chkvartheta{\check\vartheta}

\def\chkiota{\check\iota}
\def\chkkappa{\check\kappa}
\def\chklambda{\check\lambda}
\def\chkmu{\check\mu}
\def\chknu{\check\nu}
\def\chkxi{\check\xi}
\def\chkpi{\check\pi}
\def\chkrho{\check\rho}
\def\chkvarrho{\check\varrho}
\def\chksigma{\check\sigma}
\def\chkvarsigma{\check\varsigma}
\def\chktau{\check\tau}
\def\chkupsilon{\check\upsilon}
\def\chkphi{\check\phi}
\def\chkvarphi{\check\varphi}
\def\chkchi{\check\chi}
\def\chkpsi{\check\psi}
\def\chkomega{\check\omega}

\def\chkGamma{\check\Gamma}
\def\chkDelta{\check\Delta}
\def\chkTheta{\check\Theta}
\def\chkLambda{\check\Lambda}
\def\chkXi{\check\Xi}
\def\chkPi{\check\Pi}
\def\chkSigma{\check\Sigma}
\def\chkUpsilon{\check\Upsilon}
\def\chkPhi{\check\Phi}
\def\chkPsi{\check\Psi}
%%%%%%%%%%%%%%%%%%%%%%%%%%%%%%%%%%%%%%%%%%%end of check Greek
%
%
%
%%%%%%%%%%%%%%%%%%%%%%%%%%%%%%%%%%%%%%%%%%%%%%%%Bold Greek letter
%
\def\bbupsilon{{\mbox{\boldmath $\upsilon$}}}
\def\bbalpha{{\mbox{\boldmath $\alpha$}}}
\def\bbbeta{{\mbox{\boldmath $\beta$}}}
\def\bbgamma{{\mbox{\boldmath $\gamma$}}}
\def\bbdelta{{\mbox{\boldmath $\delta$}}}
\def\bbepsilon{{\mbox{\boldmath $\epsilon$}}}
\def\bbvarepsilon{{\mbox{\boldmath $\varepsilon$}}}
\def\bbzeta{{\mbox{\boldmath $\zeta$}}}
\def\bbeta{{\mbox{\boldmath $\eta$}}}
\def\bbtheta{{\mbox{\boldmath $\theta$}}}
\def\bbvartheta{{\mbox{\boldmath $\vartheta$}}}

\def\bbiota{{\mbox{\boldmath $\iota$}}}
\def\bbkappa{{\mbox{\boldmath $\kappa$}}}
\def\bblambda{{\mbox{\boldmath $\lambda$}}}
\def\bbmu{{\mbox{\boldmath $\mu$}}}
\def\bbnu{{\mbox{\boldmath $\nu$}}}
\def\bbxi{{\mbox{\boldmath $\xi$}}}
\def\bbpi{{\mbox{\boldmath $\pi$}}}
\def\bbrho{{\mbox{\boldmath $\rho$}}}
\def\bbvarrho{{\mbox{\boldmath $\varrho$}}}
\def\bbvarsigma{{\mbox{\boldmath $\varsigma$}}}
\def\bbphi{{\mbox{\boldmath $\phi$}}}
\def\bbvarphi{{\mbox{\boldmath $\varphi$}}}
\def\bbchi{{\mbox{\boldmath $\chi$}}}
\def\bbpsi{{\mbox{\boldmath $\psi$}}}
\def\bbomega{{\mbox{\boldmath $\omega$}}}

\def\bbGamma{{\mbox{\boldmath $\Gamma$}}}
\def\bbDelta{{\mbox{\boldmath $\Delta$}}}
\def\bbTheta{{\mbox{\boldmath $\Theta$}}}
\def\bbLambda{{\mbox{\boldmath $\Lambda$}}}
\def\bbXi{{\mbox{\boldmath $\Xi$}}}
\def\bbPi{{\mbox{\boldmath $\Pi$}}}
\def\bbSigma{{\mbox{\boldmath $\Sigma$}}}
\def\bbUpsilon{{\mbox{\boldmath $\Upsilon$}}}
\def\bbPhi{{\mbox{\boldmath $\Phi$}}}
\def\bbPsi{{\mbox{\boldmath $\Psi$}}}

%%%%%%%%%%%%%%%%%%%%%%%%%%%%%%%%%%%%%%%%%%%%%%%end of Bold Greek
%
%%%%%%%%%%%%%%%%%%%%%%%%%%%%%%%%%%%%%%%%%%%%%%bar Bold Greek
%
\def\bbarbbupsilon{\bar\bbupsilon}
\def\bbarbbalpha{\bar\bbalpha}
\def\bbarbbbeta{\bar\bbbeta}
\def\bbarbbgamma{\bar\bbgamma}
\def\bbarbbdelta{\bar\bbdelta}
\def\bbarbbepsilon{\bar\bbepsilon}
\def\bbarbbvarepsilon{\bar\bbvarepsilon}
\def\bbarbbzeta{\bar\bbzeta}
\def\bbarbbeta{\bar\bbeta}
\def\bbarbbtheta{\bar\bbtheta}
\def\bbarbbvartheta{\bar\bbvartheta}

\def\bbarbbiota{\bar\bbiota}
\def\bbarbbkappa{\bar\bbkappa}
\def\bbarbblambda{\bar\bblambda}
\def\bbarbbmu{\bar\bbmu}
\def\bbarbbnu{\bar\bbnu}
\def\bbarbbxi{\bar\bbxi}
\def\bbarbbpi{\bar\bbpi}
\def\bbarbbrho{\bar\bbrho}
\def\bbarbbvarrho{\bar\bbvarrho}
\def\bbarbbvarsigma{\bar\bbvarsigma}
\def\bbarbbphi{\bar\bbphi}
\def\bbarbbvarphi{\bar\bbvarphi}
\def\bbarbbchi{\bar\bbchi}
\def\bbarbbpsi{\bar\bbpsi}
\def\bbarbbomega{\bar\bbomega}

\def\bbarbbGamma{\bar\bbGamma}
\def\bbarbbDelta{\bar\bbDelta}
\def\bbarbbTheta{\bar\bbTheta}
\def\bbarbbLambda{\bar\bbLambda}
\def\bbarbbXi{\bar\bbXi}
\def\bbarbbPi{\bar\bbPi}
\def\bbarbbSigma{\bar\bbSigma}
\def\bbarbbUpsilon{\bar\bbUpsilon}
\def\bbarbbPhi{\bar\bbPhi}
\def\bbarbbPsi{\bar\bbPsi}
%%%%%%%%%%%%%%%%%%%%%%%%%%%%%%%%%%%%%%%%%%%end of bar Bold Greek
%
%%%%%%%%%%%%%%%%%%%%%%%%%%%%%%%%%%%%%%%%%%%%%%hat Bold Greek
%
\def\hhatbbupsilon{\hat\bbupsilon}
\def\hhatbbalpha{\hat\bbalpha}
\def\hhatbbbeta{\hat\bbbeta}
\def\hhatbbgamma{\hat\bbgamma}
\def\hhatbbdelta{\hat\bbdelta}
\def\hhatbbepsilon{\hat\bbepsilon}
\def\hhatbbvarepsilon{\hat\bbvarepsilon}
\def\hhatbbzeta{\hat\bbzeta}
\def\hhatbbeta{\hat\bbeta}
\def\hhatbbtheta{\hat\bbtheta}
\def\hhatbbvartheta{\hat\bbvartheta}

\def\hhatbbiota{\hat\bbiota}
\def\hhatbbkappa{\hat\bbkappa}
\def\hhatbblambda{\hat\bblambda}
\def\hhatbbmu{\hat\bbmu}
\def\hhatbbnu{\hat\bbnu}
\def\hhatbbxi{\hat\bbxi}
\def\hhatbbpi{\hat\bbpi}
\def\hhatbbrho{\hat\bbrho}
\def\hhatbbvarrho{\hat\bbvarrho}
\def\hhatbbvarsigma{\hat\bbvarsigma}
\def\hhatbbphi{\hat\bbphi}
\def\hhatbbvarphi{\hat\bbvarphi}
\def\hhatbbchi{\hat\bbchi}
\def\hhatbbpsi{\hat\bbpsi}
\def\hhatbbomega{\hat\bbomega}

\def\hhatbbGamma{\hat\bbGamma}
\def\hhatbbDelta{\hat\bbDelta}
\def\hhatbbTheta{\hat\bbTheta}
\def\hhatbbLambda{\hat\bbLambda}
\def\hhatbbXi{\hat\bbXi}
\def\hhatbbPi{\hat\bbPi}
\def\hhatbbSigma{\hat\bbSigma}
\def\hhatbbUpsilon{\hat\bbUpsilon}
\def\hhatbbPhi{\hat\bbPhi}
\def\hhatbbPsi{\hat\bbPsi}
%%%%%%%%%%%%%%%%%%%%%%%%%%%%%%%%%%%%%%%%%%%end of hat Bold Greek
%
%%%%%%%%%%%%%%%%%%%%%%%%%%%%%%%%%%%%%%%%%%%%%%tilde Bold Greek
%
\def\tdbbupsilon{\tilde\bbupsilon}
\def\tdbbalpha{\tilde\bbalpha}
\def\tdbbbeta{\tilde\bbbeta}
\def\tdbbgamma{\tilde\bbgamma}
\def\tdbbdelta{\tilde\bbdelta}
\def\tdbbepsilon{\tilde\bbepsilon}
\def\tdbbvarepsilon{\tilde\bbvarepsilon}
\def\tdbbzeta{\tilde\bbzeta}
\def\tdbbeta{\tilde\bbeta}
\def\tdbbtheta{\tilde\bbtheta}
\def\tdbbvartheta{\tilde\bbvartheta}

\def\tdbbiota{\tilde\bbiota}
\def\tdbbkappa{\tilde\bbkappa}
\def\tdbblambda{\tilde\bblambda}
\def\tdbbmu{\tilde\bbmu}
\def\tdbbnu{\tilde\bbnu}
\def\tdbbxi{\tilde\bbxi}
\def\tdbbpi{\tilde\bbpi}
\def\tdbbrho{\tilde\bbrho}
\def\tdbbvarrho{\tilde\bbvarrho}
\def\tdbbvarsigma{\tilde\bbvarsigma}
\def\tdbbphi{\tilde\bbphi}
\def\tdbbvarphi{\tilde\bbvarphi}
\def\tdbbchi{\tilde\bbchi}
\def\tdbbpsi{\tilde\bbpsi}
\def\tdbbomega{\tilde\bbomega}

\def\tdbbGamma{\tilde\bbGamma}
\def\tdbbDelta{\tilde\bbDelta}
\def\tdbbTheta{\tilde\bbTheta}
\def\tdbbLambda{\tilde\bbLambda}
\def\tdbbXi{\tilde\bbXi}
\def\tdbbPi{\tilde\bbPi}
\def\tdbbSigma{\tilde\bbSigma}
\def\tdbbUpsilon{\tilde\bbUpsilon}
\def\tdbbPhi{\tilde\bbPhi}
\def\tdbbPsi{\tilde\bbPsi}
%%%%%%%%%%%%%%%%%%%%%%%%%%%%%%%%%%%%%%%%%%%end of tilde Bold Greek
\def\deltat{\triangle t}
%%%%%%%%%%%%%%%%%%%%%%%%%%%%%%%%%%%%%%%%%%%%%%hat greek
%
\def\hhattheta{\hat\theta}
%%%%%%%%%%%%%%%%%%%%%%%%%%%%%%%%%%%%%%%%%%%end of hat greek
%
%
%
%
\def\var{{\rm var}}
\def\bbtau{{\mbox{\boldmath $\tau$}}}
%
%
%
%
%
%misc
%\def\diag{{\text{diag}}}
%\def\tr{{\text{tr}}}
%\def\tdbbs{{\tilde{\bfs}}}
%\def\tdbby{{\tilde{\bfy}}}
%\def\rank{{\text{rank}}}
%\def\bbarbbs{{\bar{\bfs}}}
%\def\bbarbbH{{\bar{\bfH}}}
%\def\bbarbby{{\bar{\bfy}}}
%\def\tdbbH{{\tilde \bfH}}
%\def\tdbbh{{\tilde \bfh}}
%\def\tdbbeta{{\tilde \bfeta}}
%\def\checkbby{{\check\bfy}}
%\def\checkbbs{{\check\bfs}}
%\def\checkbbx{{\check\bfx}}
%\def\checkbbeta{{\check\bfeta}}
%\def\checktdbbs{{\check\tdbbs}}


% Also note that the "draftcls" or "draftclsnofoot", not "draft", option
% should be used if it is desired that the figures are to be displayed in
% draft mode.
%



% *** MISC UTILITY PACKAGES ***
%
%\usepackage{ifpdf}
% Heiko Oberdiek's ifpdf.sty is very useful if you need conditional
% compilation based on whether the output is pdf or dvi.
% usage:
% \ifpdf
%   % pdf code
% \else
%   % dvi code
% \fi
% The latest version of ifpdf.sty can be obtained from:
% http://www.ctan.org/tex-archive/macros/latex/contrib/oberdiek/
% Also, note that IEEEtran.cls V1.7 and later provides a builtin
% \ifCLASSINFOpdf conditional that works the same way.
% When switching from latex to pdflatex and vice-versa, the compiler may
% have to be run twice to clear warning/error messages.






% *** CITATION PACKAGES ***
%

% cite.sty was written by Donald Arseneau
% V1.6 and later of IEEEtran pre-defines the format of the cite.sty package
% \cite{} output to follow that of IEEE. Loading the cite package will
% result in citation numbers being automatically sorted and properly
% "compressed/ranged". e.g., [1], [9], [2], [7], [5], [6] without using
% cite.sty will become [1], [2], [5]--[7], [9] using cite.sty. cite.sty's
% \cite will automatically add leading space, if needed. Use cite.sty's
% noadjust option (cite.sty V3.8 and later) if you want to turn this off.
% cite.sty is already installed on most LaTeX systems. Be sure and use
% version 4.0 (2003-05-27) and later if using hyperref.sty. cite.sty does
% not currently provide for hyperlinked citations.
% The latest version can be obtained at:
% http://www.ctan.org/tex-archive/macros/latex/contrib/cite/
% The documentation is contained in the cite.sty file itself.






% *** GRAPHICS RELATED PACKAGES ***
%
%% \ifCLASSINFOpdf
%%   % \usepackage[pdftex]{graphicx}
%%   % declare the path(s) where your graphic files are
%%   % \graphicspath{{../pdf/}{../jpeg/}}
%%   % and their extensions so you won't have to specify these with
%%   % every instance of \includegraphics
%%   % \DeclareGraphicsExtensions{.pdf,.jpeg,.png}
%% \else
%%   % or other class option (dvipsone, dvipdf, if not using dvips). graphicx
%%   % will default to the driver specified in the system graphics.cfg if no
%%   % driver is specified.
%%   % \usepackage[dvips]{graphicx}
%%   % declare the path(s) where your graphic files are
%%   % \graphicspath{{../eps/}}
%%   % and their extensions so you won't have to specify these with
%%   % every instance of \includegraphics
%%   % \DeclareGraphicsExtensions{.eps}
%% \fi

% graphicx was written by David Carlisle and Sebastian Rahtz. It is
% required if you want graphics, photos, etc. graphicx.sty is already
% installed on most LaTeX systems. The latest version and documentation can
% be obtained at: 
% http://www.ctan.org/tex-archive/macros/latex/required/graphics/
% Another good source of documentation is "Using Imported Graphics in
% LaTeX2e" by Keith Reckdahl which can be found as epslatex.ps or
% epslatex.pdf at: http://www.ctan.org/tex-archive/info/
%
% latex, and pdflatex in dvi mode, support graphics in encapsulated
% postscript (.eps) format. pdflatex in pdf mode supports graphics
% in .pdf, .jpeg, .png and .mps (metapost) formats. Users should ensure
% that all non-photo figures use a vector format (.eps, .pdf, .mps) and
% not a bitmapped formats (.jpeg, .png). IEEE frowns on bitmapped formats
% which can result in "jaggedy"/blurry rendering of lines and letters as
% well as large increases in file sizes.
%
% You can find documentation about the pdfTeX application at:
% http://www.tug.org/applications/pdftex





% *** MATH PACKAGES ***
%
%\usepackage[cmex10]{amsmath}
% A popular package from the American Mathematical Society that provides
% many useful and powerful commands for dealing with mathematics. If using
% it, be sure to load this package with the cmex10 option to ensure that
% only type 1 fonts will utilized at all point sizes. Without this option,
% it is possible that some math symbols, particularly those within
% footnotes, will be rendered in bitmap form which will result in a
% document that can not be IEEE Xplore compliant!
%
% Also, note that the amsmath package sets \interdisplaylinepenalty to 10000
% thus preventing page breaks from occurring within multiline equations. Use:
%\interdisplaylinepenalty=2500
% after loading amsmath to restore such page breaks as IEEEtran.cls normally
% does. amsmath.sty is already installed on most LaTeX systems. The latest
% version and documentation can be obtained at:
% http://www.ctan.org/tex-archive/macros/latex/required/amslatex/math/





% *** SPECIALIZED LIST PACKAGES ***
%
%\usepackage{algorithmic}
% algorithmic.sty was written by Peter Williams and Rogerio Brito.
% This package provides an algorithmic environment fo describing algorithms.
% You can use the algorithmic environment in-text or within a figure
% environment to provide for a floating algorithm. Do NOT use the algorithm
% floating environment provided by algorithm.sty (by the same authors) or
% algorithm2e.sty (by Christophe Fiorio) as IEEE does not use dedicated
% algorithm float types and packages that provide these will not provide
% correct IEEE style captions. The latest version and documentation of
% algorithmic.sty can be obtained at:
% http://www.ctan.org/tex-archive/macros/latex/contrib/algorithms/
% There is also a support site at:
% http://algorithms.berlios.de/index.html
% Also of interest may be the (relatively newer and more customizable)
% algorithmicx.sty package by Szasz Janos:
% http://www.ctan.org/tex-archive/macros/latex/contrib/algorithmicx/




% *** ALIGNMENT PACKAGES ***
%
%\usepackage{array}
% Frank Mittelbach's and David Carlisle's array.sty patches and improves
% the standard LaTeX2e array and tabular environments to provide better
% appearance and additional user controls. As the default LaTeX2e table
% generation code is lacking to the point of almost being broken with
% respect to the quality of the end results, all users are strongly
% advised to use an enhanced (at the very least that provided by array.sty)
% set of table tools. array.sty is already installed on most systems. The
% latest version and documentation can be obtained at:
% http://www.ctan.org/tex-archive/macros/latex/required/tools/


%\usepackage{mdwmath}
%\usepackage{mdwtab}
% Also highly recommended is Mark Wooding's extremely powerful MDW tools,
% especially mdwmath.sty and mdwtab.sty which are used to format equations
% and tables, respectively. The MDWtools set is already installed on most
% LaTeX systems. The lastest version and documentation is available at:
% http://www.ctan.org/tex-archive/macros/latex/contrib/mdwtools/


% IEEEtran contains the IEEEeqnarray family of commands that can be used to
% generate multiline equations as well as matrices, tables, etc., of high
% quality.


%\usepackage{eqparbox}
% Also of notable interest is Scott Pakin's eqparbox package for creating
% (automatically sized) equal width boxes - aka "natural width parboxes".
% Available at:
% http://www.ctan.org/tex-archive/macros/latex/contrib/eqparbox/





% *** SUBFIGURE PACKAGES ***
%\usepackage[tight,footnotesize]{subfigure}
% subfigure.sty was written by Steven Douglas Cochran. This package makes it
% easy to put subfigures in your figures. e.g., "Figure 1a and 1b". For IEEE
% work, it is a good idea to load it with the tight package option to reduce
% the amount of white space around the subfigures. subfigure.sty is already
% installed on most LaTeX systems. The latest version and documentation can
% be obtained at:
% http://www.ctan.org/tex-archive/obsolete/macros/latex/contrib/subfigure/
% subfigure.sty has been superceeded by subfig.sty.



%\usepackage[caption=false]{caption}
%\usepackage[font=footnotesize]{subfig}
% subfig.sty, also written by Steven Douglas Cochran, is the modern
% replacement for subfigure.sty. However, subfig.sty requires and
% automatically loads Axel Sommerfeldt's caption.sty which will override
% IEEEtran.cls handling of captions and this will result in nonIEEE style
% figure/table captions. To prevent this problem, be sure and preload
% caption.sty with its "caption=false" package option. This is will preserve
% IEEEtran.cls handing of captions. Version 1.3 (2005/06/28) and later 
% (recommended due to many improvements over 1.2) of subfig.sty supports
% the caption=false option directly:
%\usepackage[caption=false,font=footnotesize]{subfig}
%
% The latest version and documentation can be obtained at:
% http://www.ctan.org/tex-archive/macros/latex/contrib/subfig/
% The latest version and documentation of caption.sty can be obtained at:
% http://www.ctan.org/tex-archive/macros/latex/contrib/caption/




% *** FLOAT PACKAGES ***
%
\usepackage{fixltx2e}
% fixltx2e, the successor to the earlier fix2col.sty, was written by
% Frank Mittelbach and David Carlisle. This package corrects a few problems
% in the LaTeX2e kernel, the most notable of which is that in current
% LaTeX2e releases, the ordering of single and double column floats is not
% guaranteed to be preserved. Thus, an unpatched LaTeX2e can allow a
% single column figure to be placed prior to an earlier double column
% figure. The latest version and documentation can be found at:
% http://www.ctan.org/tex-archive/macros/latex/base/



%\usepackage{stfloats}
% stfloats.sty was written by Sigitas Tolusis. This package gives LaTeX2e
% the ability to do double column floats at the bottom of the page as well
% as the top. (e.g., "\begin{figure*}[!b]" is not normally possible in
% LaTeX2e). It also provides a command:
%\fnbelowfloat
% to enable the placement of footnotes below bottom floats (the standard
% LaTeX2e kernel puts them above bottom floats). This is an invasive package
% which rewrites many portions of the LaTeX2e float routines. It may not work
% with other packages that modify the LaTeX2e float routines. The latest
% version and documentation can be obtained at:
% http://www.ctan.org/tex-archive/macros/latex/contrib/sttools/
% Documentation is contained in the stfloats.sty comments as well as in the
% presfull.pdf file. Do not use the stfloats baselinefloat ability as IEEE
% does not allow \baselineskip to stretch. Authors submitting work to the
% IEEE should note that IEEE rarely uses double column equations and
% that authors should try to avoid such use. Do not be tempted to use the
% cuted.sty or midfloat.sty packages (also by Sigitas Tolusis) as IEEE does
% not format its papers in such ways.


%\ifCLASSOPTIONcaptionsoff
%  \usepackage[nomarkers]{endfloat}
% \let\MYoriglatexcaption\caption
% \renewcommand{\caption}[2][\relax]{\MYoriglatexcaption[#2]{#2}}
%\fi
% endfloat.sty was written by James Darrell McCauley and Jeff Goldberg.
% This package may be useful when used in conjunction with IEEEtran.cls'
% captionsoff option. Some IEEE journals/societies require that submissions
% have lists of figures/tables at the end of the paper and that
% figures/tables without any captions are placed on a page by themselves at
% the end of the document. If needed, the draftcls IEEEtran class option or
% \CLASSINPUTbaselinestretch interface can be used to increase the line
% spacing as well. Be sure and use the nomarkers option of endfloat to
% prevent endfloat from "marking" where the figures would have been placed
% in the text. The two hack lines of code above are a slight modification of
% that suggested by in the endfloat docs (section 8.3.1) to ensure that
% the full captions always appear in the list of figures/tables - even if
% the user used the short optional argument of \caption[]{}.
% IEEE papers do not typically make use of \caption[]'s optional argument,
% so this should not be an issue. A similar trick can be used to disable
% captions of packages such as subfig.sty that lack options to turn off
% the subcaptions:
% For subfig.sty:
% \let\MYorigsubfloat\subfloat
% \renewcommand{\subfloat}[2][\relax]{\MYorigsubfloat[]{#2}}
% For subfigure.sty:
% \let\MYorigsubfigure\subfigure
% \renewcommand{\subfigure}[2][\relax]{\MYorigsubfigure[]{#2}}
% However, the above trick will not work if both optional arguments of
% the \subfloat/subfig command are used. Furthermore, there needs to be a
% description of each subfigure *somewhere* and endfloat does not add
% subfigure captions to its list of figures. Thus, the best approach is to
% avoid the use of subfigure captions (many IEEE journals avoid them anyway)
% and instead reference/explain all the subfigures within the main caption.
% The latest version of endfloat.sty and its documentation can obtained at:
% http://www.ctan.org/tex-archive/macros/latex/contrib/endfloat/
%
% The IEEEtran \ifCLASSOPTIONcaptionsoff conditional can also be used
% later in the document, say, to conditionally put the References on a 
% page by themselves.





% *** PDF, URL AND HYPERLINK PACKAGES ***
%
%\usepackage{url}
% url.sty was written by Donald Arseneau. It provides better support for
% handling and breaking URLs. url.sty is already installed on most LaTeX
% systems. The latest version can be obtained at:
% http://www.ctan.org/tex-archive/macros/latex/contrib/misc/
% Read the url.sty source comments for usage information. Basically,
% \url{my_url_here}.





% *** Do not adjust lengths that control margins, column widths, etc. ***
% *** Do not use packages that alter fonts (such as pslatex).         ***
% There should be no need to do such things with IEEEtran.cls V1.6 and later.
% (Unless specifically asked to do so by the journal or conference you plan
% to submit to, of course. )



% An example of a floating figure using the graphicx package.
% Note that \label must occur AFTER (or within) \caption.
% For figures, \caption should occur after the \includegraphics.
% Note that IEEEtran v1.7 and later has special internal code that
% is designed to preserve the operation of \label within \caption
% even when the captionsoff option is in effect. However, because
% of issues like this, it may be the safest practice to put all your
% \label just after \caption rather than within \caption{}.
%
% Reminder: the "draftcls" or "draftclsnofoot", not "draft", class
% option should be used if it is desired that the figures are to be
% displayed while in draft mode.
%
%\begin{figure}[!t]
%\centering
%\includegraphics[width=2.5in]{myfigure}
% where an .eps filename suffix will be assumed under latex, 
% and a .pdf suffix will be assumed for pdflatex; or what has been declared
% via \DeclareGraphicsExtensions.
%\caption{Simulation Results}
%\label{fig_sim}
%\end{figure}

% Note that IEEE typically puts floats only at the top, even when this
% results in a large percentage of a column being occupied by floats.


% An example of a double column floating figure using two subfigures.
% (The subfig.sty package must be loaded for this to work.)
% The subfigure \label commands are set within each subfloat command, the
% \label for the overall figure must come after \caption.
% \hfil must be used as a separator to get equal spacing.
% The subfigure.sty package works much the same way, except \subfigure is
% used instead of \subfloat.
%
%\begin{figure*}[!t]
%\centerline{\subfloat[Case I]\includegraphics[width=2.5in]{subfigcase1}%
%\label{fig_first_case}}
%\hfil
%\subfloat[Case II]{\includegraphics[width=2.5in]{subfigcase2}%
%\label{fig_second_case}}}
%\caption{Simulation results}
%\label{fig_sim}
%\end{figure*}
%
% Note that often IEEE papers with subfigures do not employ subfigure
% captions (using the optional argument to \subfloat), but instead will
% reference/describe all of them (a), (b), etc., within the main caption.


% An example of a floating table. Note that, for IEEE style tables, the 
% \caption command should come BEFORE the table. Table text will default to
% \footnotesize as IEEE normally uses this smaller font for tables.
% The \label must come after \caption as always.
%
%\begin{table}[!t]
%% increase table row spacing, adjust to taste
%\renewcommand{\arraystretch}{1.3}
% if using array.sty, it might be a good idea to tweak the value of
% \extrarowheight as needed to properly center the text within the cells
%\caption{An Example of a Table}
%\label{table_example}
%\centering
%% Some packages, such as MDW tools, offer better commands for making tables
%% than the plain LaTeX2e tabular which is used here.
%\begin{tabular}{|c||c|}
%\hline
%One & Two\\
%\hline
%Three & Four\\
%\hline
%\end{tabular}
%\end{table}


% Note that IEEE does not put floats in the very first column - or typically
% anywhere on the first page for that matter. Also, in-text middle ("here")
% positioning is not used. Most IEEE journals use top floats exclusively.
% Note that, LaTeX2e, unlike IEEE journals, places footnotes above bottom
% floats. This can be corrected via the \fnbelowfloat command of the
% stfloats package.


% trigger a \newpage just before the given reference
% number - used to balance the columns on the last page
% adjust value as needed - may need to be readjusted if
% the document is modified later
%\IEEEtriggeratref{8}
% The "triggered" command can be changed if desired:
%\IEEEtriggercmd{\enlargethispage{-5in}}

% references section

% can use a bibliography generated by BibTeX as a .bbl file
% BibTeX documentation can be easily obtained at:
% http://www.ctan.org/tex-archive/biblio/bibtex/contrib/doc/
% The IEEEtran BibTeX style support page is at:
% http://www.michaelshell.org/tex/ieeetran/bibtex/
%\bibliographystyle{IEEEtran}
% argument is your BibTeX string definitions and bibliography database(s)
%\bibliography{IEEEabrv,../bib/paper}
%
% <OR> manually copy in the resultant .bbl file
% set second argument of \begin to the number of references
% (used to reserve space for the reference number labels box)




%% biography section
%% % 
%% % If you have an EPS/PDF photo (graphicx package needed) extra braces are
%% % needed around the contents of the optional argument to biography to prevent
%% % the LaTeX parser from getting confused when it sees the complicated
%% % \includegraphics command within an optional argument. (You could create
%% % your own custom macro containing the \includegraphics command to make things
%% % simpler here.)
%% %\begin{biography}[{\includegraphics[width=1in,height=1.25in,clip,keepaspectratio]{mshell}}]{Michael Shell}
%% % or if you just want to reserve a space for a photo:

%% \begin{IEEEbiography}{Michael Shell}
%% Biography text here.
%% \end{IEEEbiography}

%% % if you will not have a photo at all:
%% \begin{IEEEbiographynophoto}{John Doe}
%% Biography text here.
%% \end{IEEEbiographynophoto}

%% % insert where needed to balance the two columns on the last page with
%% % biographies
%% %\newpage

%% \begin{IEEEbiographynophoto}{Jane Doe}
%% Biography text here.
%% \end{IEEEbiographynophoto}

%% % You can push biographies down or up by placing
%% % a \vfill before or after them. The appropriate
%% % use of \vfill depends on what kind of text is
%% % on the last page and whether or not the columns
%% % are being equalized.

%% %\vfill

%% % Can be used to pull up biographies so that the bottom of the last one
%% % is flush with the other column.
%% %\enlargethispage{-5in}



\usepackage{graphicx}

%\usepackage{subfig}  % I have read that subcaption works better than subfig
\usepackage{subcaption}              %subfigures with \mbox and \subfigure[]

\usepackage[utf8]{inputenc}


%\usepackage[spanish]{babel}
%\usepackage[galician]{babel}

\usepackage{amsfonts}
\usepackage{amsmath}
\usepackage{mathtools}
%\usepackage{amsthm}   % use \qedhere to place the QED 
                      % symbol at the end of a math line
                      % or environment {proof}.

\usepackage{amssymb}

\usepackage{bbold}
\usepackage{bm}
%\usepackage{cancel} % use \cancel{expression}, \xcancel{}, etc

\usepackage{color,verbatim}
\usepackage{multirow}
\usepackage{accents}

\usepackage{theoremref}
% use \thlabel{} and \thref{}

\usepackage{algorithm}
%\usepackage[options ]{algorithm2e}
\usepackage[noend]{algpseudocode}
\floatname{algorithm}{Procedure}


%\usepackage{balance}  % if you want to balance the height of the two columns
% at the last page write \balance on the first column of the last page.

%\usepackage{flushend}

\usepackage{url}

%\usepackage{lineno} %-> you should copy the .sty files to the current
% folder and include \linenumbers after \begin{document}. Note, the
% package does not work well with environment *equation*. It is better
% to use the environment *eqnarray* for equations.


%%%%%%%%%%%%%%%%%%%%%%%%%%%%%%%%%%%%%%%%%%%%%%%%%%%%
%% CUSTOMIZATION %%%%%%%%%%%%%%%%%%%%%%%%%%%%%%%%%%%
%%%%%%%%%%%%%%%%%%%%%%%%%%%%%%%%%%%%%%%%%%%%%%%%%%%%

%\def\figurename{Figura}
%\def\refname{Referencias}


%% \topmargin 0truein
%% \topskip 0truein
%% \headheight 0truein
%% \headsep 0.5truein
%% %\footheight 0truein
%% \oddsidemargin 0.0in
%% \evensidemargin 0.0in
%% \textwidth 6.5in
%% \textheight 8.5in

%%%%%%%%%%%%%%%%%%%%%%%%%%%%%%%%%%%%%%%%%%%%%%%%%%%%
%% ENVIRONMENTS%%%%%%%%%%%%%%%%%%%%%%%%%%%%%%%%%%%%%
%%%%%%%%%%%%%%%%%%%%%%%%%%%%%%%%%%%%%%%%%%%%%%%%%%%%

\newenvironment{mycolorexample}
        { \begin{colbox}{orange}
\textit{Example:}\\ \small    }
        {   \end{colbox}  }

\newenvironment{myexample}
        {\textit{Example:}\\ \small    }
        {     }

\newenvironment{mycolorinformalproof}
        {% \begin{colbox}{gray}
 \small\color{darkblue}    \textit{Informal Proof:}\\}
        { % \end{colbox}  
}

\newenvironment{mycolorproof}
        {% \begin{colbox}{gray}
 \small\color{darkblue}    \textit{Proof:}\\}
        { % \end{colbox}  
}


\newenvironment{mycolortheorem}
        { \begin{colbox}{lightred}
\color{black}    \begin{mytheorem}}
        {\end{mytheorem}
 \end{colbox}  
}

\newenvironment{mycolorlemma}
        { \begin{colbox}{lightgray}
\color{black}    \begin{mylemma}}
        {\end{mylemma}
 \end{colbox}  
}



\newenvironment{mycolordefinition}
        { \begin{colbox}{lightgreen}
\color{black}    \begin{mydefinition}}
        {\end{mydefinition}
 \end{colbox}  
}


\newenvironment{mycolorproposition}
        { \begin{colbox}{lightgray}
\color{black}    \begin{myproposition}}
        {\end{myproposition}
 \end{colbox}  
}

\newenvironment{mycolorcorollary}
        { \begin{colbox}{lightgray}
\color{black}    \begin{mycorollary}}
        {\end{mycorollary}
 \end{colbox}  
}




\newenvironment{mycolormnemonic}
        { \begin{colbox}{lightblue}
\color{black}    \textbf{Mnemonic:}}
        {
 \end{colbox}  
}



\newenvironment{mycolorremark}
        { \begin{colbox}{lightgray}
\color{black}    \begin{myremark}}
        {\end{myremark}
 \end{colbox}  
}



\newenvironment{mycolorproblemstatement}
        { \begin{colbox}{purple}
\color{black}    \begin{myproblemstatement}}
        {\end{myproblemstatement}
 \end{colbox}  
}



\newcommand{\mycolorbox}[2]{
 \begin{colbox}{purple}
\color{black}    
\textbf{#1}\\
#2
 \end{colbox}  
}





\newcounter{rulecounter}
\newcommand{\resetrule}{ \setcounter{rulecounter}{0}}
\resetrule
\newenvironment{ruleenv}[1]
               {\refstepcounter{rulecounter}
               \par\noindent \textbf{Rule \arabic{rulecounter}:} \textit{#1}\\}
               {  }





\newenvironment{algoritm}[1]
               {\vspace{0.3in}
                 \begin{center}
                   \parbox{6in}{#1}
                 \end{center}
                 \vspace{0.15in}
                 \hrule
                 \begin{enumerate}}
               {
                 \end{enumerate}
                 \hrule
                 \vspace{0.3in}}

\newenvironment{exercise}[1]
               {\par\noindent\underline{\textbf{Exercise #1}}\bf\par\noindent}
               {}

\newenvironment{question}
               {\par\noindent \begin{itshape}\par\noindent\begin{itemize}\item \bf}
               {\end{itemize}\end{itshape}}

\newenvironment{mysolution}
               {\par\noindent\underline{{Solution}}\par\indent}
               {\begin{flushright} $\Box$ \end{flushright}}


% Color box 
\newsavebox{\selvestebox}
\newenvironment{colbox}[1]
  {\newcommand\colboxcolor{#1}%
   \begin{lrbox}{\selvestebox}%
   \begin{minipage}{\dimexpr\columnwidth-2\fboxsep\relax}}
  {\end{minipage}\end{lrbox}%
   \begin{center}
   \colorbox{\colboxcolor}{\usebox{\selvestebox}}
   \end{center}}

% Colors
\definecolor{orange}{rgb}{1,0.8,0}
\definecolor{gray}{rgb}{.9,0.9,0.9}
\definecolor{darkgray}{rgb}{.3,0.3,0.3}
\definecolor{darkblue}{rgb}{.1,0.0,0.3}
\definecolor{lightblue}{rgb}{0.7,0.7,1}
\definecolor{lightred}{rgb}{1,0.7,.7}
\definecolor{purple}{RGB}{204,153,255}
\definecolor{lightgray}{rgb}{.95,0.95,0.95}
\definecolor{lightgreen}{rgb}{0.3,0.5,0.3}
\definecolor{darkgreen}{rgb}{0.05,0.3,0.05}



%%%%%%%%%%%%%%%%%%%%%%%%%%%%%%%%%%%%%%%%%%%%%%%%%%%%
%% COMMANDS %%%%%%%%%%%%%%%%%%%%%%%%%%%%%%%%%%%%%%%%
%%%%%%%%%%%%%%%%%%%%%%%%%%%%%%%%%%%%%%%%%%%%%%%%%%%%

% text mode
\newcommand{\vs}{\emph{vs.}}
\newcommand{\ea}{\emph{et al.}}
\newcommand{\ra}{$\rightarrow$~}
\newcommand{\Ra}{$\Rightarrow$~}

% math mode
\newcommand{\fourierpair}{~\xleftrightarrow{~\mathcal{F}~}~}
\newcommand{\ubar}[1]{\underaccent{\bar}{#1}}
\newcommand{\ubbm}[1]{\ubar{\bm{#1}}}
\newcommand{\brackets}[1]{\left\{#1\right\}}
%\newcommand{\q}[1]{\mbox{$q^{- #1}$}}
\newcommand{\bbm}[1]{{\bar{\bm #1}}}
\newcommand{\dbbm}[1]{{\dbar{\bm #1}}}
\newcommand{\dbar}[1]{{\bar{\bar{ #1}}}}
\newcommand{\tbm}[1]{{\tilde{\bm #1}}}
\newcommand{\cbm}[1]{{\check{\bm #1}}}
\newcommand{\hbm}[1]{{\hat{\bm #1}}}
\newcommand{\inv}{^{-1}}
\newcommand{\pinv}{^{\dagger}}
\newcommand{\msqrt}{^{1/2}}
\newcommand{\msqrtinv}{^{-1/2}}

\newcommand{\ffield}{\mathbb{F}}
\newcommand{\cfield}{\mathbb{C}}
\newcommand{\rfield}{\mathbb{R}}
\newcommand{\vect}{\mathop{\rm vec}}
\newcommand{\vectinv}{\mathop{\rm vec}^{-1}}
\newcommand{\re}[1]{\mathop{\rm Re}\brackets{#1}}
\newcommand{\im}[1]{\mathop{\rm Im}\brackets{#1}}
\newcommand{\cov}[2]{\mathop{\rm Cov}\left\{#1,#2\right\}}
\newcommand{\diag}[1]{\mathop{\rm diag}\brackets{#1}}
\newcommand{\toep}[1]{\mathop{\rm toep}\brackets{#1}}
\newcommand{\bdiag}[1]{\mathop{\rm bdiag}\brackets{#1}}
\newcommand{\diagnb}{\mathop{\rm diag}}
\newcommand{\vectnb}{\mathop{\rm vec}}
\newcommand{\rankb}[1]{\mathop{\rm rank}\brackets{#1}}
\newcommand{\ranknb}{\mathop{\rm rank}}
\newcommand{\rank}{\mathop{\rm rank}}
\newcommand{\krank}{\mathop{\rm krank}}
\newcommand{\lambdamax}{\lambda_\text{max}}
\newcommand{\lambdamin}{\lambda_\text{mmin}}
\newcommand{\dimf}{\mathop{\rm dim}_\ffield}
\newcommand{\dimr}{\mathop{\rm dim}_\rfield}
\newcommand{\dimc}{\mathop{\rm dim}_\cfield}
\newcommand{\spanv}{\mathop{\rm span}}
\newcommand{\spanvf}{\spanv_{\ffield}}
\newcommand{\spanvr}{\spanv_{\rfield}}
\newcommand{\spanvc}{\spanv_{\cfield}}
\newcommand{\vecv}{\mathop{\rm vec}}
\newcommand{\defect}[1]{\mathop{\rm def}\brackets{#1}}
\newcommand{\range}[1]{\mathcal{R}\brackets{#1}}
\newcommand{\nullspace}[1]{\mathcal{N}\brackets{#1}}
%\newcommand{\snr}{\mathop{\rm SNR}}
\newcommand{\tr}[1]{\mathop{\rm Tr}\left(#1\right)}
\newcommand{\trnb}{\mathop{\rm Tr}}
%\newcommand{\etr}[1]{\mathop{\mathrm{Etr}}\left\{#1\right\}}
\newcommand{\etr}[1]{\operatorname{etr}\left\{#1\right\}}
\newcommand{\etrnb}{\operatorname{etr}}
\newcommand{\trs}[1]{\mathop{\rm Tr^2}\left(#1\right)}
\newcommand{\sign}[1]{\mathop{\rm sign}\brackets{#1}}
\newcommand{\GM}{\mathop{\rm GM}}
\newcommand{\rect}{\mathop{\rm rect}}
\def\QED{~\rule[-1pt]{5pt}{5pt}\par\medskip}
\newcommand{\sinc}{\mathop{\rm sinc}}
\newcommand{\transpose}{^T}
 \newcommand{\define}{\triangleq}
% \newcommand{\conv}{\star} % convolution
\newcommand{\conv}{{\hspace{3pt}\star}\hspace{3pt}} % convolution

\newcommand{\equalls}{
  \begin{array}{c}
    \text{\tiny{\text{LS}}} \\[-0.5cm]
    = \\[-0.3cm]
    ~
  \end{array}
}

\newcommand{\intinfty}{\int_{-\infty}^\infty}

\newcommand{\zmo}{z^{-1}}
\newcommand{\ellon}{$\ell^1$-norm}
\newcommand{\elltn}{$\ell^2$-norm}

\newcommand{\prob}[1]{\mathop{\textrm{P}} \brackets{#1} }
\newcommand{\probnb}{\mathop{\textrm{P}} }
\newcommand{\expected}[1]{\mathop{\textrm{E}}\brackets{#1} }
\newcommand{\expectednb}{\mathop{\textrm{E}}\nolimits}
\newcommand{\expectedwrt}[2]{\mathop{\textrm{E}}\nolimits_{#1}\{#2\}}
\newcommand{\expb}[1]{\exp \brackets{#1} }
\newcommand{\var}[1]{\mathop{\textrm{Var}}\brackets{#1} }

\newcommand{\pd}{P_\text{D}}
\newcommand{\pfa}{P_\text{FA}}
\newcommand{\hz}{{\mathcal{H}_0}}
\newcommand{\ho}{{\mathcal{H}_1}}
\newcommand{\hi}{{\mathcal{H}_i}}
\newcommand{\decide}{\mathop{\gtrless}^{\mathcal{H}_1}_{\mathcal{H}_0}}
\newcommand{\decideinv}{\mathop{\gtrless}^{\mathcal{H}_0}_{\mathcal{H}_1}}
\newcommand{\minimize}{\mathop{\text{minimize}}}
\newcommand{\maximize}{\mathop{\text{maximize}}}
\newcommand{\optimize}{\mathop{\text{optimize}}}
\newcommand{\st}{\mathop{\text{s.t.}}}
\newcommand{\subjectto}{\mathop{\text{subject to}}}
\newcommand{\image}{\mathop{\text{image}}}
\newcommand{\ejo}{(e^{j\omega})}
\DeclareMathOperator*{\argmin}{arg\,min}
\DeclareMathOperator*{\argmax}{arg\,max}


%%%%%%%%%%%%%%%%%%%%%%%%%%%%%%%%%%%%%%%%%%%%%%%%%%%%

\newtheorem{myproposition}{Proposition}
\newtheorem{myremark}{Remark}
\newtheorem{myproblemstatement}{Problem Statement}
\newtheorem{mylemma}{Lemma}
\newtheorem{mytheorem}{Theorem}
\newtheorem{mydefinition}{Definition}
\newtheorem{mycorollary}{Corollary}




%%%%%%%%%%%%%%%%%%%%%%%%%%%%%%%%%%%%%%%%%%%%%%%%%%%%
%% TEMPLATES %%%%%%%%%%%%%%%%%%%%%%%%%%%%%%%%%%%%%%%
%%%%%%%%%%%%%%%%%%%%%%%%%%%%%%%%%%%%%%%%%%%%%%%%%%%%

%%
%% COUNTERS:
%%
%% \addtocounter{counter}{value} increments counter by 
%%     the amount, which can be negative
%% \alph{counter}, \Alph{counter} print the value of
%%      the counter as a lower or upper case letter.
%% \arabic{counter} print the value of the counter as an 
%%      arabic number \fnsymbol{counter} print the counter 
%%      as a footnote symbol 
%% \newcounter define a new counter 
%% \roman{counter}, \Roman{counter} print the value of the
%%     counter as a roman letter using lower or upper case
%%     letters 
%% \setcounter{counter}{value} assign the value to the
%%     counter 
%% \usecounter{counter} to be used in list environment.  
%% \value{counter} get the value of the counter.
%% \refstepcounter{counter} increments and makes label 
%%     command take its value


%%   Illustrate functions over sets
%%    
%%    \begin{align}
%%      \begin{array}{ccccc}
%%        V&\multicolumn{3}{c}{\xrightarrow{\hspace{2cm}f\hspace{2cm}}} &\mathbb{R}\\
%%        V&\xrightarrow{~f_1~}&W&\xrightarrow{~f_2~} &\mathbb{R}\\
%%        v&\rightarrow&w=f_1(v)&\rightarrow &f_2(w)
%%      \end{array}
%%    \end{align}



%% 
% you can use \xleftarrow{something} to display an 
% arrow with text above.
%%


%%%%%%%%%%%%%%%%%%%%%%%%%%%%%%%%%%%%%%%%%%%%%%%%%%%%
% % % % % % % % % % % % % % % % % % % % % % % % % % 
%\abstract{}

%\begin{exercise}{1: Exercise name}
%\end{exercise}

%\begin{question}
%\end{question}

%\begin{solution}
%\end{solution}

%\clearpage

%\displaystyle : in arrays of equations, placed before the expression,
%                it prevents it from becoming smaller.

%\begin{figure}[bhtp]
%  \centering
%  \includegraphics[width=1.0\textwidth]{route.png or route.jpg}
%  \caption{    }
%  \label{  }
%\end{figure}


%% \begin{figure}
%% \begin{minipage}[b]{.5\linewidth}
%% \centering\large A
%% \subcaption{A subfigure}\label{fig:1a}
%% \end{minipage}%
%% \begin{minipage}[b]{.5\linewidth}
%% \centering\large B
%% \subcaption{Another subfigure}\label{fig:1b}
%% \end{minipage}
%% \caption{A figure}\label{fig:1}
%% \end{figure}


%% \begin{figure}[bhtp]
%%  \centering
%%  \subfloat[one]{
%%    \label{}
%%    \includegraphics[width=1.0\textwidth]{route.png or route.jpg}
%%  } %\\ -> to arrange vertically
%%  \subfloat[two]{
%%    \label{}
%%    \includegraphics[width=1.0\textwidth]{route.png or route.jpg}
%%  } % etc
%%  \caption{    }
%%  \label{  }
%% \end{figure}


% to add a figure from xfig export it as *Combined PS/LaTeX (both
% parts)* (=> extension .pstex). Then substitute
% \includegraphics... in this file by \input{filename.pstex_t}


%   \begin{table}[bhtp] % table as image
%     \centering
%     \includegraphics[width=1.0\textwidth]{route.png etc}
%     \caption{    }
%     \label{ }
%   \end{table}

% To include multiline equations in tables use, inside a cell:
%% \begin{center}
%%    $\begin{aligned}
%%    exp1 & exp2 \\
%%    exp3 & exp4 \\
%%    \end{aligned}$
%% \end{center}

% aligned also allows to use \boxed{} for multi-line equations

%% z = \overbrace{
%%     \underbrace{x}_\text{real} +
%%     \underbrace{iy}_\text{imaginary}
%%    }^\text{complex number}

%%  \renewcommand{\refname}{} % to remove the word ``REFERENCES''
%%  from the bibliography.


%% This is to attach a letter to the reference numbers, e.g. [2],[33]  by [P2],[P33]
%% \makeatletter
%% \renewcommand*{\@biblabel}[1]{[P#1]}
%% \makeatother
%% \renewcommand\citeform[1]{P#1}




%% \usepackage{algorithmic}
%% \usepackage[ruled,vlined,resetcount]{algorithm2e}
%
%% \begin{algorithm}                   
%% \caption{Censored Algorithm}
%% \label{algo:1}    
%% \begin{minipage}{200cm}
%% \begin{algorithmic}[1]
%% \STATE Initialize 
%% \FOR{$n=1,2,\ldots$}
%% \IF{uuu}
%% \STATE Estimate
%% \ENDIF
%% \STATE Set 
%% \ENDFOR
%% \STATE Set 
%% \end{algorithmic}
%% \end{minipage}
%% \end{algorithm}



% \begin{center}
%   \begin{tabular}{|c | c | c|}
%     \hline
%      &  & \\
%     \hline
%     \hline
%      &  & \\
%     \hline
%     &  & \\
%     \hline
%     &  & \\
%     \hline
%   \end{tabular}
% \end{center}

%% use \renewcommand{\arraystretch}{2} before to make rows  higher



  
% \appendix     % the following sections will be numerated as appendices

% \begin{thebibliography}{10}
% \bibitem{ labelx } Author, eth  % for cite: \cite{ labelx }
% \end{thebibliography}

% text in two columns:
% \begin{tabular}{p{5cm} p{5cm}}
% paragraph1  &  paragraph2\\
% \end{tabular}


% EQUATIONS IN TWO COLUMNS:
%% \begin{align}
%% \begin{aligned}
%%   \minimize &x+y+z\\
%% \st&a_1\leq x\leq b_1\\
%% &a_2\leq y+z\leq b_2
%% \end{aligned}
%% \hspace{3cm}
%% \begin{aligned}
%%   \minimize &x+y+z\\
%% \st&a_1\leq x\leq b_1\\
%% &a_2\leq y+z\leq b_2
%% \end{aligned}
%% \end{align}

%% COLOR BOXES
%% \begin{colbox}{red}
%% TEXT, EQUATIONS, ETC
%% \end{colbox}

%% FUNCTIONS IN EMACS TO COMPILE

%% A) BIBTEX + PDFLATEX:
%%  (defun compileltx ( )
%% (interactive)
%% (progn
%%  (save-buffer) 
%%  (setq name (buffer-name))

%%   (progn 
%%     (setq posdot (string-match "\\." name ))
%%     (setq pdfname (concat (substring name 0  posdot) ".pdf" ) ) 

%%  (setq posdot (string-match "\\." name ))
%%  (setq auxname (concat (substring name 0  posdot) ".aux" ) )

%% (call-process "bibtex" nil "bibtex"  nil auxname) 

%% (setq outvalue (call-process "pdflatex" nil "pdflatex"  nil name) )
%% (if (> outvalue 0) (switch-to-buffer "pdflatex")
%%   (progn 
%%     (setq posdot (string-match "\\." name ))
%%     (setq pdfname (concat (substring name 0  posdot) ".pdf" ) )
%% ; (call-process "evincescript.sh" nil  nil nil pdfname) 
%%     (start-process "my-process" "foo" "evince" pdfname))
%% )
%% )
%% )
%% )


%% (global-set-key "\C-c\C-w" 'compileltx)

% unset case insensitive search
%% (setq case-fold-search nil)




%
%%%%%%%%%%%%%%%%%%%%%%%%%%%%%%%%%theorem environments
%
\newtheorem{assumption}{\hspace{0pt}\bf AS\hspace{-0.15cm}}

\newcounter{lemma} \setcounter{lemma}{1}
\newenvironment{lemma}{\vspace{0.1cm} \noindent {\bf Lemma
\thelemma :} \addtocounter{lemma}{-1}\refstepcounter{lemma} \em} {\addtocounter{lemma}{1}
 \normalfont }


\newcounter{proposition} \setcounter{proposition}{1}
\newenvironment{proposition}{\vspace{0.1cm} \noindent {\bf Proposition
\theproposition :} \addtocounter{proposition}{-1}\refstepcounter{proposition}\em } { \addtocounter{proposition}{1}
 \normalfont }

\newcounter{corollary} \setcounter{corollary}{1}
\newenvironment{corollary}{\vspace{0.1cm} \noindent {\bf Corollary
\thecorollary :} \addtocounter{corollary}{-1}\refstepcounter{corollary}\em } { \addtocounter{corollary}{1}
 \normalfont }




\newtheorem{observation}{\hspace{0pt}\bf Observation}
\newtheorem{theorem}{\hspace{0pt}\bf Theorem}
%\newtheorem{corollary}{\hspace{0pt}\bf Corollary}
\newtheorem{fact}{\hspace{0pt}\bf Fact}

%\newcounter{remark} \setcounter{remark}{1}
%\newenvironment{remark}{\vspace{-0.1cm} \noindent {\bf Remark
%\theremark :} \addtocounter{remark}{-1} \refstepcounter{remark}} {
%\vspace{-0.1cm} \normalfont \addtocounter{remark}{1} }

\newtheorem{remark}{\hspace{-11pt}\bf Remark}
%\newtheorem{remark}{\noindent \bf Remark}

\newtheorem{test}{\hspace{0pt}\it Test Case}
\newtheorem{definition}{\hspace{0pt}\bf Definition}
\newtheorem{example}{\hspace{0pt}\bf Example}

\def\proof      {\noindent\hspace{0pt}{\bf{Proof: }}}
\def\myQED      {\hfill$\square$\vspace{0.3cm}}








%%%%%%%%%%%%%%%%%%%%%%%%%%%%%%%%%bar version
%capital alphabet
\def\bbarA{{\ensuremath{\bar A}}}
\def\bbarB{{\ensuremath{\bar B}}}
\def\bbarC{{\ensuremath{\bar C}}}
\def\bbarD{{\ensuremath{\bar D}}}
\def\bbarE{{\ensuremath{\bar E}}}
\def\bbarF{{\ensuremath{\bar F}}}
\def\bbarG{{\ensuremath{\bar G}}}
\def\bbarH{{\ensuremath{\bar H}}}
\def\bbarI{{\ensuremath{\bar I}}}
\def\bbarJ{{\ensuremath{\bar J}}}
\def\bbarK{{\ensuremath{\bar K}}}
\def\bbarL{{\ensuremath{\bar L}}}
\def\bbarM{{\ensuremath{\bar M}}}
\def\bbarN{{\ensuremath{\bar N}}}
\def\bbarO{{\ensuremath{\bar O}}}
\def\bbarP{{\ensuremath{\bar P}}}
\def\bbarQ{{\ensuremath{\bar Q}}}
\def\bbarR{{\ensuremath{\bar R}}}
\def\bbarW{{\ensuremath{\bar W}}}
\def\bbarU{{\ensuremath{\bar U}}}
\def\bbarV{{\ensuremath{\bar V}}}
\def\bbarS{{\ensuremath{\bar S}}}
\def\bbarT{{\ensuremath{\bar T}}}
\def\bbarX{{\ensuremath{\bar X}}}
\def\bbarY{{\ensuremath{\bar Y}}}
\def\bbarZ{{\ensuremath{\bar Z}}}
%lower case alphabet
\def\bbara{{\ensuremath{\bar a}}}
\def\bbarb{{\ensuremath{\bar b}}}
\def\bbarc{{\ensuremath{\bar c}}}
\def\bbard{{\ensuremath{\bar d}}}
\def\bbare{{\ensuremath{\bar e}}}
\def\bbarf{{\ensuremath{\bar f}}}
\def\bbarg{{\ensuremath{\bar g}}}
\def\bbarh{{\ensuremath{\bar h}}}
\def\bbari{{\ensuremath{\bar i}}}
\def\bbarj{{\ensuremath{\bar j}}}
\def\bbark{{\ensuremath{\bar k}}}
\def\bbarl{{\ensuremath{\bar l}}}
\def\bbarm{{\ensuremath{\bar m}}}
\def\bbarn{{\ensuremath{\bar n}}}
\def\bbaro{{\ensuremath{\bar o}}}
\def\bbarp{{\ensuremath{\bar p}}}
\def\bbarq{{\ensuremath{\bar q}}}
\def\bbarr{{\ensuremath{\bar r}}}
\def\bbarw{{\ensuremath{\bar w}}}
\def\bbaru{{\ensuremath{\bar u}}}
\def\bbarv{{\ensuremath{\bar v}}}
\def\bbars{{\ensuremath{\bar s}}}
\def\bbart{{\ensuremath{\bar t}}}
\def\bbarx{{\ensuremath{\bar x}}}
\def\bbary{{\ensuremath{\bar y}}}
\def\bbarz{{\ensuremath{\bar z}}}
%%%%%%%%%%%%%%%%%%%%%%%%%%%%%%%%%%end of bar version
%
%%%%%%%%%%%%%%%%%%%%%%%%%%%%%%%%%%%%%caligraph version
\def\ccalA{{\ensuremath{\mathcal A}}}
\def\ccalB{{\ensuremath{\mathcal B}}}
\def\ccalC{{\ensuremath{\mathcal C}}}
\def\ccalD{{\ensuremath{\mathcal D}}}
\def\ccalE{{\ensuremath{\mathcal E}}}
\def\ccalF{{\ensuremath{\mathcal F}}}
\def\ccalG{{\ensuremath{\mathcal G}}}
\def\ccalH{{\ensuremath{\mathcal H}}}
\def\ccalI{{\ensuremath{\mathcal I}}}
\def\ccalJ{{\ensuremath{\mathcal J}}}
\def\ccalK{{\ensuremath{\mathcal K}}}
\def\ccalL{{\ensuremath{\mathcal L}}}
\def\ccalM{{\ensuremath{\mathcal M}}}
\def\ccalN{{\ensuremath{\mathcal N}}}
\def\ccalO{{\ensuremath{\mathcal O}}}
\def\ccalP{{\ensuremath{\mathcal P}}}
\def\ccalQ{{\ensuremath{\mathcal Q}}}
\def\ccalR{{\ensuremath{\mathcal R}}}
\def\ccalW{{\ensuremath{\mathcal W}}}
\def\ccalU{{\ensuremath{\mathcal U}}}
\def\ccalV{{\ensuremath{\mathcal V}}}
\def\ccalS{{\ensuremath{\mathcal S}}}
\def\ccalT{{\ensuremath{\mathcal T}}}
\def\ccalX{{\ensuremath{\mathcal X}}}
\def\ccalY{{\ensuremath{\mathcal Y}}}
\def\ccalZ{{\ensuremath{\mathcal Z}}}
%lower case alphabet
\def\ccala{{\ensuremath{\mathcal a}}}
\def\ccalb{{\ensuremath{\mathcal b}}}
\def\ccalc{{\ensuremath{\mathcal c}}}
\def\ccald{{\ensuremath{\mathcal d}}}
\def\ccale{{\ensuremath{\mathcal e}}}
\def\ccalf{{\ensuremath{\mathcal f}}}
\def\ccalg{{\ensuremath{\mathcal g}}}
\def\ccalh{{\ensuremath{\mathcal h}}}
\def\ccali{{\ensuremath{\mathcal i}}}
\def\ccalj{{\ensuremath{\mathcal j}}}
\def\ccalk{{\ensuremath{\mathcal k}}}
\def\ccall{{\ensuremath{\mathcal l}}}
\def\ccalm{{\ensuremath{\mathcal m}}}
\def\ccaln{{\ensuremath{\mathcal n}}}
\def\ccalo{{\ensuremath{\mathcal o}}}
\def\ccalp{{\ensuremath{\mathcal p}}}
\def\ccalq{{\ensuremath{\mathcal q}}}
\def\ccalr{{\ensuremath{\mathcal r}}}
\def\ccalw{{\ensuremath{\mathcal w}}}
\def\ccalu{{\ensuremath{\mathcal u}}}
\def\ccalv{{\ensuremath{\mathcal v}}}
\def\ccals{{\ensuremath{\mathcal s}}}
\def\ccalt{{\ensuremath{\mathcal t}}}
\def\ccalx{{\ensuremath{\mathcal x}}}
\def\ccaly{{\ensuremath{\mathcal y}}}
\def\ccalz{{\ensuremath{\mathcal z}}}
\def\ccal0{{\ensuremath{\mathcal 0}}}
%%%%%%%%%%%%%%%%%%%%%%%%%%%%%%%%%%%%%%%%%end of caligraph version
%
%
%%%%%%%%%%%%%%%%%%%%%%%%%%%%%%%%%%%%%%%%%%%hat version
%capital alphabet
\def\hhatA{{\ensuremath{\hat A}}}
\def\hhatB{{\ensuremath{\hat B}}}
\def\hhatC{{\ensuremath{\hat C}}}
\def\hhatD{{\ensuremath{\hat D}}}
\def\hhatE{{\ensuremath{\hat E}}}
\def\hhatF{{\ensuremath{\hat F}}}
\def\hhatG{{\ensuremath{\hat G}}}
\def\hhatH{{\ensuremath{\hat H}}}
\def\hhatI{{\ensuremath{\hat I}}}
\def\hhatJ{{\ensuremath{\hat J}}}
\def\hhatK{{\ensuremath{\hat K}}}
\def\hhatL{{\ensuremath{\hat L}}}
\def\hhatM{{\ensuremath{\hat M}}}
\def\hhatN{{\ensuremath{\hat N}}}
\def\hhatO{{\ensuremath{\hat O}}}
\def\hhatP{{\ensuremath{\hat P}}}
\def\hhatQ{{\ensuremath{\hat Q}}}
\def\hhatR{{\ensuremath{\hat R}}}
\def\hhatW{{\ensuremath{\hat W}}}
\def\hhatU{{\ensuremath{\hat U}}}
\def\hhatV{{\ensuremath{\hat V}}}
\def\hhatS{{\ensuremath{\hat S}}}
\def\hhatT{{\ensuremath{\hat T}}}
\def\hhatX{{\ensuremath{\hat X}}}
\def\hhatY{{\ensuremath{\hat Y}}}
\def\hhatZ{{\ensuremath{\hat Z}}}
%lower case alphabet
\def\hhata{{\ensuremath{\hat a}}}
\def\hhatb{{\ensuremath{\hat b}}}
\def\hhatc{{\ensuremath{\hat c}}}
\def\hhatd{{\ensuremath{\hat d}}}
\def\hhate{{\ensuremath{\hat e}}}
\def\hhatf{{\ensuremath{\hat f}}}
\def\hhatg{{\ensuremath{\hat g}}}
\def\hhath{{\ensuremath{\hat h}}}
\def\hhati{{\ensuremath{\hat i}}}
\def\hhatj{{\ensuremath{\hat j}}}
\def\hhatk{{\ensuremath{\hat k}}}
\def\hhatl{{\ensuremath{\hat l}}}
\def\hhatm{{\ensuremath{\hat m}}}
\def\hhatn{{\ensuremath{\hat n}}}
\def\hhato{{\ensuremath{\hat o}}}
\def\hhatp{{\ensuremath{\hat p}}}
\def\hhatq{{\ensuremath{\hat q}}}
\def\hhatr{{\ensuremath{\hat r}}}
\def\hhatw{{\ensuremath{\hat w}}}
\def\hhatu{{\ensuremath{\hat u}}}
\def\hhatv{{\ensuremath{\hat v}}}
\def\hhats{{\ensuremath{\hat s}}}
\def\hhatt{{\ensuremath{\hat t}}}
\def\hhatx{{\ensuremath{\hat x}}}
\def\hhaty{{\ensuremath{\hat y}}}
\def\hhatz{{\ensuremath{\hat z}}}
%%%%%%%%%%%%%%%%%%%%%%%%%%%%%%%%%%end of hat version
%
%
%%%%%%%%%%%%%%%%%%%%%%%%%%%%%%%%%%tilde version
%capital alphabet
\def\tdA{{\ensuremath{\tilde A}}}
\def\tdB{{\ensuremath{\tilde B}}}
\def\tdC{{\ensuremath{\tilde C}}}
\def\tdD{{\ensuremath{\tilde D}}}
\def\tdE{{\ensuremath{\tilde E}}}
\def\tdF{{\ensuremath{\tilde F}}}
\def\tdG{{\ensuremath{\tilde G}}}
\def\tdH{{\ensuremath{\tilde H}}}
\def\tdI{{\ensuremath{\tilde I}}}
\def\tdJ{{\ensuremath{\tilde J}}}
\def\tdK{{\ensuremath{\tilde K}}}
\def\tdL{{\ensuremath{\tilde L}}}
\def\tdM{{\ensuremath{\tilde M}}}
\def\tdN{{\ensuremath{\tilde N}}}
\def\tdO{{\ensuremath{\tilde O}}}
\def\tdP{{\ensuremath{\tilde P}}}
\def\tdQ{{\ensuremath{\tilde Q}}}
\def\tdR{{\ensuremath{\tilde R}}}
\def\tdW{{\ensuremath{\tilde W}}}
\def\tdU{{\ensuremath{\tilde U}}}
\def\tdV{{\ensuremath{\tilde V}}}
\def\tdS{{\ensuremath{\tilde S}}}
\def\tdT{{\ensuremath{\tilde T}}}
\def\tdX{{\ensuremath{\tilde X}}}
\def\tdY{{\ensuremath{\tilde Y}}}
\def\tdZ{{\ensuremath{\tilde Z}}}
%lower case alphabet
\def\tda{{\ensuremath{\tilde a}}}
\def\tdb{{\ensuremath{\tilde b}}}
\def\tdc{{\ensuremath{\tilde c}}}
\def\tdd{{\ensuremath{\tilde d}}}
\def\tde{{\ensuremath{\tilde e}}}
\def\tdf{{\ensuremath{\tilde f}}}
\def\tdg{{\ensuremath{\tilde g}}}
\def\tdh{{\ensuremath{\tilde h}}}
\def\tdi{{\ensuremath{\tilde i}}}
\def\tdj{{\ensuremath{\tilde j}}}
\def\tdk{{\ensuremath{\tilde k}}}
\def\tdl{{\ensuremath{\tilde l}}}
\def\tdm{{\ensuremath{\tilde m}}}
\def\tdn{{\ensuremath{\tilde n}}}
\def\tdo{{\ensuremath{\tilde o}}}
\def\tdp{{\ensuremath{\tilde p}}}
\def\tdq{{\ensuremath{\tilde q}}}
\def\tdr{{\ensuremath{\tilde r}}}
\def\tdw{{\ensuremath{\tilde w}}}
\def\tdu{{\ensuremath{\tilde u}}}
\def\tdv{{\ensuremath{\tilde r}}}
\def\tds{{\ensuremath{\tilde s}}}
\def\tdt{{\ensuremath{\tilde t}}}
\def\tdx{{\ensuremath{\tilde x}}}
\def\tdy{{\ensuremath{\tilde y}}}
\def\tdz{{\ensuremath{\tilde z}}}
%%%%%%%%%%%%%%%%%%%%%%%%%%%%%%%%%%%%end of tilde version
%
%%%%%%%%%%%%%%%%%%%%%%%%%%%%%%%%%%%%%check version
%lower case alphabet
\def\chka{{\ensuremath{\check a}}}
\def\chkb{{\ensuremath{\check b}}}
\def\chkc{{\ensuremath{\check c}}}
\def\chkd{{\ensuremath{\check d}}}
\def\chke{{\ensuremath{\check e}}}
\def\chkf{{\ensuremath{\check f}}}
\def\chkg{{\ensuremath{\check g}}}
\def\chkh{{\ensuremath{\check h}}}
\def\chki{{\ensuremath{\check i}}}
\def\chkj{{\ensuremath{\check j}}}
\def\chkk{{\ensuremath{\check k}}}
\def\chkl{{\ensuremath{\check l}}}
\def\chkm{{\ensuremath{\check m}}}
\def\chkn{{\ensuremath{\check n}}}
\def\chko{{\ensuremath{\check o}}}
\def\chkp{{\ensuremath{\check p}}}
\def\chkq{{\ensuremath{\check q}}}
\def\chkr{{\ensuremath{\check r}}}
\def\chkw{{\ensuremath{\check w}}}
\def\chku{{\ensuremath{\check u}}}
\def\chkv{{\ensuremath{\check v}}}
\def\chks{{\ensuremath{\check s}}}
\def\chkt{{\ensuremath{\check t}}}
\def\chkx{{\ensuremath{\check x}}}
\def\chky{{\ensuremath{\check y}}}
\def\chkz{{\ensuremath{\check z}}}
%%%%%%%%%%%%%%%%%%%%%%%%%%%%%%%%%%end of check version
%
%
%%%%%%%%%%%%%%%%%%%%%%%%%%%%%%%%%%%%Bold version
% upper case bold
\def\bbA{{\ensuremath{\mathbf A}}}
\def\bbB{{\ensuremath{\mathbf B}}}
\def\bbC{{\ensuremath{\mathbf C}}}
\def\bbD{{\ensuremath{\mathbf D}}}
\def\bbE{{\ensuremath{\mathbf E}}}
\def\bbF{{\ensuremath{\mathbf F}}}
\def\bbG{{\ensuremath{\mathbf G}}}
\def\bbH{{\ensuremath{\mathbf H}}}
\def\bbI{{\ensuremath{\mathbf I}}}
\def\bbJ{{\ensuremath{\mathbf J}}}
\def\bbK{{\ensuremath{\mathbf K}}}
\def\bbL{{\ensuremath{\mathbf L}}}
\def\bbM{{\ensuremath{\mathbf M}}}
\def\bbN{{\ensuremath{\mathbf N}}}
\def\bbO{{\ensuremath{\mathbf O}}}
\def\bbP{{\ensuremath{\mathbf P}}}
\def\bbQ{{\ensuremath{\mathbf Q}}}
\def\bbR{{\ensuremath{\mathbf R}}}
\def\bbW{{\ensuremath{\mathbf W}}}
\def\bbU{{\ensuremath{\mathbf U}}}
\def\bbV{{\ensuremath{\mathbf V}}}
\def\bbS{{\ensuremath{\mathbf S}}}
\def\bbT{{\ensuremath{\mathbf T}}}
\def\bbX{{\ensuremath{\mathbf X}}}
\def\bbY{{\ensuremath{\mathbf Y}}}
\def\bbZ{{\ensuremath{\mathbf Z}}}
%lower case bold
\def\bba{{\ensuremath{\mathbf a}}}
\def\bbb{{\ensuremath{\mathbf b}}}
\def\bbc{{\ensuremath{\mathbf c}}}
\def\bbd{{\ensuremath{\mathbf d}}}
\def\bbe{{\ensuremath{\mathbf e}}}
\def\bbf{{\ensuremath{\mathbf f}}}
\def\bbg{{\ensuremath{\mathbf g}}}
\def\bbh{{\ensuremath{\mathbf h}}}
\def\bbi{{\ensuremath{\mathbf i}}}
\def\bbj{{\ensuremath{\mathbf j}}}
\def\bbk{{\ensuremath{\mathbf k}}}
\def\bbl{{\ensuremath{\mathbf l}}}
\def\bbm{{\ensuremath{\mathbf m}}}
\def\bbn{{\ensuremath{\mathbf n}}}
\def\bbo{{\ensuremath{\mathbf o}}}
\def\bbp{{\ensuremath{\mathbf p}}}
\def\bbq{{\ensuremath{\mathbf q}}}
\def\bbr{{\ensuremath{\mathbf r}}}
\def\bbw{{\ensuremath{\mathbf w}}}
\def\bbu{{\ensuremath{\mathbf u}}}
\def\bbv{{\ensuremath{\mathbf v}}}
\def\bbs{{\ensuremath{\mathbf s}}}
\def\bbt{{\ensuremath{\mathbf t}}}
\def\bbx{{\ensuremath{\mathbf x}}}
\def\bby{{\ensuremath{\mathbf y}}}
\def\bbz{{\ensuremath{\mathbf z}}}
\def\bb0{{\ensuremath{\mathbf 0}}}
%

%%%%%%%%%%%%%%%%%%%%%%%%%%%%%%%%%%%%%%%%%%%%%%bar bold version
%upper case
%
\def\bbarbbA{{\bar{\ensuremath{\mathbf A}} }}
\def\bbarbbB{{\bar{\ensuremath{\mathbf B}} }}
\def\bbarbbC{{\bar{\ensuremath{\mathbf C}} }}
\def\bbarbbD{{\bar{\ensuremath{\mathbf D}} }}
\def\bbarbbE{{\bar{\ensuremath{\mathbf E}} }}
\def\bbarbbF{{\bar{\ensuremath{\mathbf F}} }}
\def\bbarbbG{{\bar{\ensuremath{\mathbf G}} }}
\def\bbarbbH{{\bar{\ensuremath{\mathbf H}} }}
\def\bbarbbI{{\bar{\ensuremath{\mathbf I}} }}
\def\bbarbbJ{{\bar{\ensuremath{\mathbf J}} }}
\def\bbarbbK{{\bar{\ensuremath{\mathbf K}} }}
\def\bbarbbL{{\bar{\ensuremath{\mathbf L}} }}
\def\bbarbbM{{\bar{\ensuremath{\mathbf M}} }}
\def\bbarbbN{{\bar{\ensuremath{\mathbf N}} }}
\def\bbarbbO{{\bar{\ensuremath{\mathbf O}} }}
\def\bbarbbP{{\bar{\ensuremath{\mathbf P}} }}
\def\bbarbbQ{{\bar{\ensuremath{\mathbf Q}} }}
\def\bbarbbR{{\bar{\ensuremath{\mathbf R}} }}
\def\bbarbbS{{\bar{\ensuremath{\mathbf S}} }}
\def\bbarbbT{{\bar{\ensuremath{\mathbf T}} }}
\def\bbarbbU{{\bar{\ensuremath{\mathbf U}} }}
\def\bbarbbV{{\bar{\ensuremath{\mathbf V}} }}
\def\bbarbbW{{\bar{\ensuremath{\mathbf W}} }}
\def\bbarbbX{{\bar{\ensuremath{\mathbf X}} }}
\def\bbarbbY{{\bar{\ensuremath{\mathbf Y}} }}
\def\bbarbbZ{{\bar{\ensuremath{\mathbf Z}} }}
%
%lower case
%
\def\bbarbba{{\bar{\ensuremath{\mathbf a}} }}
\def\bbarbbb{{\bar{\ensuremath{\mathbf b}} }}
\def\bbarbbc{{\bar{\ensuremath{\mathbf c}} }}
\def\bbarbbd{{\bar{\ensuremath{\mathbf d}} }}
\def\bbarbbe{{\bar{\ensuremath{\mathbf e}} }}
\def\bbarbbf{{\bar{\ensuremath{\mathbf f}} }}
\def\bbarbbg{{\bar{\ensuremath{\mathbf g}} }}
\def\bbarbbh{{\bar{\ensuremath{\mathbf h}} }}
\def\bbarbbi{{\bar{\ensuremath{\mathbf i}} }}
\def\bbarbbj{{\bar{\ensuremath{\mathbf j}} }}
\def\bbarbbk{{\bar{\ensuremath{\mathbf k}} }}
\def\bbarbbl{{\bar{\ensuremath{\mathbf l}} }}
\def\bbarbbm{{\bar{\ensuremath{\mathbf m}} }}
\def\bbarbbn{{\bar{\ensuremath{\mathbf n}} }}
\def\bbarbbo{{\bar{\ensuremath{\mathbf o}} }}
\def\bbarbbp{{\bar{\ensuremath{\mathbf p}} }}
\def\bbarbbq{{\bar{\ensuremath{\mathbf q}} }}
\def\bbarbbr{{\bar{\ensuremath{\mathbf r}} }}
\def\bbarbbs{{\bar{\ensuremath{\mathbf s}} }}
\def\bbarbbt{{\bar{\ensuremath{\mathbf t}} }}
\def\bbarbbu{{\bar{\ensuremath{\mathbf u}} }}
\def\bbarbbv{{\bar{\ensuremath{\mathbf v}} }}
\def\bbarbbw{{\bar{\ensuremath{\mathbf w}} }}
\def\bbarbbx{{\bar{\ensuremath{\mathbf x}} }}
\def\bbarbby{{\bar{\ensuremath{\mathbf y}} }}
\def\bbarbbz{{\bar{\ensuremath{\mathbf z}} }}
%%%%%%%%%%%%%%%%%%%%%%%%%%%%%%%%%%%%%%%%%%%%%%%end of bar bold bersion
%
%
%%%%%%%%%%%%%%%%%%%%%%%%%%%%%%%%%%%%%%%%%%%%%%hat bold version
%upper case
%
\def\hhatbbA{{\hat{\ensuremath{\mathbf A}} }}
\def\hhatbbB{{\hat{\ensuremath{\mathbf B}} }}
\def\hhatbbC{{\hat{\ensuremath{\mathbf C}} }}
\def\hhatbbD{{\hat{\ensuremath{\mathbf D}} }}
\def\hhatbbE{{\hat{\ensuremath{\mathbf E}} }}
\def\hhatbbF{{\hat{\ensuremath{\mathbf F}} }}
\def\hhatbbG{{\hat{\ensuremath{\mathbf G}} }}
\def\hhatbbH{{\hat{\ensuremath{\mathbf H}} }}
\def\hhatbbI{{\hat{\ensuremath{\mathbf I}} }}
\def\hhatbbJ{{\hat{\ensuremath{\mathbf J}} }}
\def\hhatbbK{{\hat{\ensuremath{\mathbf K}} }}
\def\hhatbbL{{\hat{\ensuremath{\mathbf L}} }}
\def\hhatbbM{{\hat{\ensuremath{\mathbf M}} }}
\def\hhatbbN{{\hat{\ensuremath{\mathbf N}} }}
\def\hhatbbO{{\hat{\ensuremath{\mathbf O}} }}
\def\hhatbbP{{\hat{\ensuremath{\mathbf P}} }}
\def\hhatbbQ{{\hat{\ensuremath{\mathbf Q}} }}
\def\hhatbbR{{\hat{\ensuremath{\mathbf R}} }}
\def\hhatbbS{{\hat{\ensuremath{\mathbf S}} }}
\def\hhatbbT{{\hat{\ensuremath{\mathbf T}} }}
\def\hhatbbU{{\hat{\ensuremath{\mathbf U}} }}
\def\hhatbbV{{\hat{\ensuremath{\mathbf V}} }}
\def\hhatbbW{{\hat{\ensuremath{\mathbf W}} }}
\def\hhatbbX{{\hat{\ensuremath{\mathbf X}} }}
\def\hhatbbY{{\hat{\ensuremath{\mathbf Y}} }}
\def\hhatbbZ{{\hat{\ensuremath{\mathbf Z}} }}
%
%lower case
%
\def\hhatbba{{\hat{\ensuremath{\mathbf a}} }}
\def\hhatbbb{{\hat{\ensuremath{\mathbf b}} }}
\def\hhatbbc{{\hat{\ensuremath{\mathbf c}} }}
\def\hhatbbd{{\hat{\ensuremath{\mathbf d}} }}
\def\hhatbbe{{\hat{\ensuremath{\mathbf e}} }}
\def\hhatbbf{{\hat{\ensuremath{\mathbf f}} }}
\def\hhatbbg{{\hat{\ensuremath{\mathbf g}} }}
\def\hhatbbh{{\hat{\ensuremath{\mathbf h}} }}
\def\hhatbbi{{\hat{\ensuremath{\mathbf i}} }}
\def\hhatbbj{{\hat{\ensuremath{\mathbf j}} }}
\def\hhatbbk{{\hat{\ensuremath{\mathbf k}} }}
\def\hhatbbl{{\hat{\ensuremath{\mathbf l}} }}
\def\hhatbbm{{\hat{\ensuremath{\mathbf m}} }}
\def\hhatbbn{{\hat{\ensuremath{\mathbf n}} }}
\def\hhatbbo{{\hat{\ensuremath{\mathbf o}} }}
\def\hhatbbp{{\hat{\ensuremath{\mathbf p}} }}
\def\hhatbbq{{\hat{\ensuremath{\mathbf q}} }}
\def\hhatbbr{{\hat{\ensuremath{\mathbf r}} }}
\def\hhatbbs{{\hat{\ensuremath{\mathbf s}} }}
\def\hhatbbt{{\hat{\ensuremath{\mathbf t}} }}
\def\hhatbbu{{\hat{\ensuremath{\mathbf u}} }}
\def\hhatbbv{{\hat{\ensuremath{\mathbf v}} }}
\def\hhatbbw{{\hat{\ensuremath{\mathbf w}} }}
\def\hhatbbx{{\hat{\ensuremath{\mathbf x}} }}
\def\hhatbby{{\hat{\ensuremath{\mathbf y}} }}
\def\hhatbbz{{\hat{\ensuremath{\mathbf z}} }}
%%%%%%%%%%%%%%%%%%%%%%%%%%%%%%%%%%%%%%%%%%%%%%%end of hat bold  bersion
%
%
%%%%%%%%%%%%%%%%%%%%%%%%%%%%%%%%%%%%%%%%%%%%%%tilde bold version
%upper case
%
\def\tdbbA{{\tilde{\ensuremath{\mathbf A}} }}
\def\tdbbB{{\tilde{\ensuremath{\mathbf B}} }}
\def\tdbbC{{\tilde{\ensuremath{\mathbf C}} }}
\def\tdbbD{{\tilde{\ensuremath{\mathbf D}} }}
\def\tdbbE{{\tilde{\ensuremath{\mathbf E}} }}
\def\tdbbF{{\tilde{\ensuremath{\mathbf F}} }}
\def\tdbbG{{\tilde{\ensuremath{\mathbf G}} }}
\def\tdbbH{{\tilde{\ensuremath{\mathbf H}} }}
\def\tdbbI{{\tilde{\ensuremath{\mathbf I}} }}
\def\tdbbJ{{\tilde{\ensuremath{\mathbf J}} }}
\def\tdbbK{{\tilde{\ensuremath{\mathbf K}} }}
\def\tdbbL{{\tilde{\ensuremath{\mathbf L}} }}
\def\tdbbM{{\tilde{\ensuremath{\mathbf M}} }}
\def\tdbbN{{\tilde{\ensuremath{\mathbf N}} }}
\def\tdbbO{{\tilde{\ensuremath{\mathbf O}} }}
\def\tdbbP{{\tilde{\ensuremath{\mathbf P}} }}
\def\tdbbQ{{\tilde{\ensuremath{\mathbf Q}} }}
\def\tdbbR{{\tilde{\ensuremath{\mathbf R}} }}
\def\tdbbS{{\tilde{\ensuremath{\mathbf S}} }}
\def\tdbbT{{\tilde{\ensuremath{\mathbf T}} }}
\def\tdbbU{{\tilde{\ensuremath{\mathbf U}} }}
\def\tdbbV{{\tilde{\ensuremath{\mathbf V}} }}
\def\tdbbW{{\tilde{\ensuremath{\mathbf W}} }}
\def\tdbbX{{\tilde{\ensuremath{\mathbf X}} }}
\def\tdbbY{{\tilde{\ensuremath{\mathbf Y}} }}
\def\tdbbZ{{\tilde{\ensuremath{\mathbf Z}} }}
%
%lower case
%
\def\tdbba{{\tilde{\ensuremath{\mathbf a}} }}
\def\tdbbb{{\tilde{\ensuremath{\mathbf b}} }}
\def\tdbbc{{\tilde{\ensuremath{\mathbf c}} }}
\def\tdbbd{{\tilde{\ensuremath{\mathbf d}} }}
\def\tdbbe{{\tilde{\ensuremath{\mathbf e}} }}
\def\tdbbf{{\tilde{\ensuremath{\mathbf f}} }}
\def\tdbbg{{\tilde{\ensuremath{\mathbf g}} }}
\def\tdbbh{{\tilde{\ensuremath{\mathbf h}} }}
\def\tdbbi{{\tilde{\ensuremath{\mathbf i}} }}
\def\tdbbj{{\tilde{\ensuremath{\mathbf j}} }}
\def\tdbbk{{\tilde{\ensuremath{\mathbf k}} }}
\def\tdbbl{{\tilde{\ensuremath{\mathbf l}} }}
\def\tdbbm{{\tilde{\ensuremath{\mathbf m}} }}
\def\tdbbn{{\tilde{\ensuremath{\mathbf n}} }}
\def\tdbbo{{\tilde{\ensuremath{\mathbf o}} }}
\def\tdbbp{{\tilde{\ensuremath{\mathbf p}} }}
\def\tdbbq{{\tilde{\ensuremath{\mathbf q}} }}
\def\tdbbr{{\tilde{\ensuremath{\mathbf r}} }}
\def\tdbbs{{\tilde{\ensuremath{\mathbf s}} }}
\def\tdbbt{{\tilde{\ensuremath{\mathbf t}} }}
\def\tdbbu{{\tilde{\ensuremath{\mathbf u}} }}
\def\tdbbv{{\tilde{\ensuremath{\mathbf v}} }}
\def\tdbbw{{\tilde{\ensuremath{\mathbf w}} }}
\def\tdbbx{{\tilde{\ensuremath{\mathbf x}} }}
\def\tdbby{{\tilde{\ensuremath{\mathbf y}} }}
\def\tdbbz{{\tilde{\ensuremath{\mathbf z}} }}
%%%%%%%%%%%%%%%%%%%%%%%%%%%%%%%%%%%%%%%%%%%%%%%end of tilde bold  bersion
%
%%%%%%%%%%%%%%%%%%%%%%%%%%%%%%%%%%%%%%%%%%%%%%%bold caligraph
%
\def\bbcalA{\mbox{\boldmath $\mathcal{A}$}}
\def\bbcalB{\mbox{\boldmath $\mathcal{B}$}}
\def\bbcalC{\mbox{\boldmath $\mathcal{C}$}}
\def\bbcalD{\mbox{\boldmath $\mathcal{D}$}}
\def\bbcalE{\mbox{\boldmath $\mathcal{E}$}}
\def\bbcalF{\mbox{\boldmath $\mathcal{F}$}}
\def\bbcalG{\mbox{\boldmath $\mathcal{G}$}}
\def\bbcalH{\mbox{\boldmath $\mathcal{H}$}}
\def\bbcalI{\mbox{\boldmath $\mathcal{I}$}}
\def\bbcalJ{\mbox{\boldmath $\mathcal{J}$}}
\def\bbcalK{\mbox{\boldmath $\mathcal{K}$}}
\def\bbcalL{\mbox{\boldmath $\mathcal{L}$}}
\def\bbcalM{\mbox{\boldmath $\mathcal{M}$}}
\def\bbcalN{\mbox{\boldmath $\mathcal{N}$}}
\def\bbcalO{\mbox{\boldmath $\mathcal{O}$}}
\def\bbcalP{\mbox{\boldmath $\mathcal{P}$}}
\def\bbcalQ{\mbox{\boldmath $\mathcal{Q}$}}
\def\bbcalR{\mbox{\boldmath $\mathcal{R}$}}
\def\bbcalW{\mbox{\boldmath $\mathcal{W}$}}
\def\bbcalU{\mbox{\boldmath $\mathcal{U}$}}
\def\bbcalV{\mbox{\boldmath $\mathcal{V}$}}
\def\bbcalS{\mbox{\boldmath $\mathcal{S}$}}
\def\bbcalT{\mbox{\boldmath $\mathcal{T}$}}
\def\bbcalX{\mbox{\boldmath $\mathcal{X}$}}
\def\bbcalY{\mbox{\boldmath $\mathcal{Y}$}}
\def\bbcalZ{\mbox{\boldmath $\mathcal{Z}$}}
%
%%%%%%%%%%%%%%%%%%%%%%%%%%%%%%%%%%%%%%%%%%%%%%%%%%end of caligraph
%
%
%
%
%%%%%%%%%%%%%%%%%%%%%%%%%%%%%%%%%%%%%%%%%%%%%%%tilde Greek
%
\def\tdupsilon{\tilde\upsilon}
\def\tdalpha{\tilde\alpha}
\def\tdbeta{\tilde\beta}
\def\tdgamma{\tilde\gamma}
\def\tddelta{\tilde\delta}
\def\tdepsilon{\tilde\epsilon}
\def\tdvarepsilon{\tilde\varepsilon}
\def\tdzeta{\tilde\zeta}
\def\tdeta{\tilde\eta}
\def\tdtheta{\tilde\theta}
\def\tdvartheta{\tilde\vartheta}

\def\tdiota{\tilde\iota}
\def\tdkappa{\tilde\kappa}
\def\tdlambda{\tilde\lambda}
\def\tdmu{\tilde\mu}
\def\tdnu{\tilde\nu}
\def\tdxi{\tilde\xi}
\def\tdpi{\tilde\pi}
\def\tdrho{\tilde\rho}
\def\tdvarrho{\tilde\varrho}
\def\tdsigma{\tilde\sigma}
\def\tdvarsigma{\tilde\varsigma}
\def\tdtau{\tilde\tau}
\def\tdupsilon{\tilde\upsilon}
\def\tdphi{\tilde\phi}
\def\tdvarphi{\tilde\varphi}
\def\tdchi{\tilde\chi}
\def\tdpsi{\tilde\psi}
\def\tdomega{\tilde\omega}

\def\tdGamma{\tilde\Gamma}
\def\tdDelta{\tilde\Delta}
\def\tdTheta{\tilde\Theta}
\def\tdLambda{\tilde\Lambda}
\def\tdXi{\tilde\Xi}
\def\tdPi{\tilde\Pi}
\def\tdSigma{\tilde\Sigma}
\def\tdUpsilon{\tilde\Upsilon}
\def\tdPhi{\tilde\Phi}
\def\tdPsi{\tilde\Psi}
%%%%%%%%%%%%%%%%%%%%%%%%%%%%%%%%%%%%%%%%%%%end of title  Greek
%
%%%%%%%%%%%%%%%%%%%%%%%%%%%%%%%%%%%%%%%%%%%%%%%bar Greek
%
\def\bbarupsilon{\bar\upsilon}
\def\bbaralpha{\bar\alpha}
\def\bbarbeta{\bar\beta}
\def\bbargamma{\bar\gamma}
\def\bbardelta{\bar\delta}
\def\bbarepsilon{\bar\epsilon}
\def\bbarvarepsilon{\bar\varepsilon}
\def\bbarzeta{\bar\zeta}
\def\bbareta{\bar\eta}
\def\bbartheta{\bar\theta}
\def\bbarvartheta{\bar\vartheta}

\def\bbariota{\bar\iota}
\def\bbarkappa{\bar\kappa}
\def\bbarlambda{\bar\lambda}
\def\bbarmu{\bar\mu}
\def\bbarnu{\bar\nu}
\def\bbarxi{\bar\xi}
\def\bbarpi{\bar\pi}
\def\bbarrho{\bar\rho}
\def\bbarvarrho{\bar\varrho}
\def\bbarvarsigma{\bar\varsigma}
\def\bbarphi{\bar\phi}
\def\bbarvarphi{\bar\varphi}
\def\bbarchi{\bar\chi}
\def\bbarpsi{\bar\psi}
\def\bbaromega{\bar\omega}

\def\bbarGamma{\bar\Gamma}
\def\bbarDelta{\bar\Delta}
\def\bbarTheta{\bar\Theta}
\def\bbarLambda{\bar\Lambda}
\def\bbarXi{\bar\Xi}
\def\bbarPi{\bar\Pi}
\def\bbarSigma{\bar\Sigma}
\def\bbarUpsilon{\bar\Upsilon}
\def\bbarPhi{\bar\Phi}
\def\bbarPsi{\bar\Psi}
%%%%%%%%%%%%%%%%%%%%%%%%%%%%%%%%%%%%%%%%%%%end of bar  Greek
%
%
%
%%%%%%%%%%%%%%%%%%%%%%%%%%%%%%%%%%%%%%%%%%%%%%%begion of check Greek
%
\def\chkupsilon{\check\upsilon}
\def\chkalpha{\check\alpha}
\def\chkbeta{\check\beta}
\def\chkgamma{\check\gamma}
\def\chkdelta{\check\delta}
\def\chkepsilon{\check\epsilon}
\def\chkvarepsilon{\check\varepsilon}
\def\chkzeta{\check\zeta}
\def\chketa{\check\eta}
\def\chktheta{\check\theta}
\def\chkvartheta{\check\vartheta}

\def\chkiota{\check\iota}
\def\chkkappa{\check\kappa}
\def\chklambda{\check\lambda}
\def\chkmu{\check\mu}
\def\chknu{\check\nu}
\def\chkxi{\check\xi}
\def\chkpi{\check\pi}
\def\chkrho{\check\rho}
\def\chkvarrho{\check\varrho}
\def\chksigma{\check\sigma}
\def\chkvarsigma{\check\varsigma}
\def\chktau{\check\tau}
\def\chkupsilon{\check\upsilon}
\def\chkphi{\check\phi}
\def\chkvarphi{\check\varphi}
\def\chkchi{\check\chi}
\def\chkpsi{\check\psi}
\def\chkomega{\check\omega}

\def\chkGamma{\check\Gamma}
\def\chkDelta{\check\Delta}
\def\chkTheta{\check\Theta}
\def\chkLambda{\check\Lambda}
\def\chkXi{\check\Xi}
\def\chkPi{\check\Pi}
\def\chkSigma{\check\Sigma}
\def\chkUpsilon{\check\Upsilon}
\def\chkPhi{\check\Phi}
\def\chkPsi{\check\Psi}
%%%%%%%%%%%%%%%%%%%%%%%%%%%%%%%%%%%%%%%%%%%end of check Greek
%
%
%
%%%%%%%%%%%%%%%%%%%%%%%%%%%%%%%%%%%%%%%%%%%%%%%%Bold Greek letter
%
\def\bbupsilon{{\mbox{\boldmath $\upsilon$}}}
\def\bbalpha{{\mbox{\boldmath $\alpha$}}}
\def\bbbeta{{\mbox{\boldmath $\beta$}}}
\def\bbgamma{{\mbox{\boldmath $\gamma$}}}
\def\bbdelta{{\mbox{\boldmath $\delta$}}}
\def\bbepsilon{{\mbox{\boldmath $\epsilon$}}}
\def\bbvarepsilon{{\mbox{\boldmath $\varepsilon$}}}
\def\bbzeta{{\mbox{\boldmath $\zeta$}}}
\def\bbeta{{\mbox{\boldmath $\eta$}}}
\def\bbtheta{{\mbox{\boldmath $\theta$}}}
\def\bbvartheta{{\mbox{\boldmath $\vartheta$}}}

\def\bbiota{{\mbox{\boldmath $\iota$}}}
\def\bbkappa{{\mbox{\boldmath $\kappa$}}}
\def\bblambda{{\mbox{\boldmath $\lambda$}}}
\def\bbmu{{\mbox{\boldmath $\mu$}}}
\def\bbnu{{\mbox{\boldmath $\nu$}}}
\def\bbxi{{\mbox{\boldmath $\xi$}}}
\def\bbpi{{\mbox{\boldmath $\pi$}}}
\def\bbrho{{\mbox{\boldmath $\rho$}}}
\def\bbvarrho{{\mbox{\boldmath $\varrho$}}}
\def\bbvarsigma{{\mbox{\boldmath $\varsigma$}}}
\def\bbphi{{\mbox{\boldmath $\phi$}}}
\def\bbvarphi{{\mbox{\boldmath $\varphi$}}}
\def\bbchi{{\mbox{\boldmath $\chi$}}}
\def\bbpsi{{\mbox{\boldmath $\psi$}}}
\def\bbomega{{\mbox{\boldmath $\omega$}}}

\def\bbGamma{{\mbox{\boldmath $\Gamma$}}}
\def\bbDelta{{\mbox{\boldmath $\Delta$}}}
\def\bbTheta{{\mbox{\boldmath $\Theta$}}}
\def\bbLambda{{\mbox{\boldmath $\Lambda$}}}
\def\bbXi{{\mbox{\boldmath $\Xi$}}}
\def\bbPi{{\mbox{\boldmath $\Pi$}}}
\def\bbSigma{{\mbox{\boldmath $\Sigma$}}}
\def\bbUpsilon{{\mbox{\boldmath $\Upsilon$}}}
\def\bbPhi{{\mbox{\boldmath $\Phi$}}}
\def\bbPsi{{\mbox{\boldmath $\Psi$}}}

%%%%%%%%%%%%%%%%%%%%%%%%%%%%%%%%%%%%%%%%%%%%%%%end of Bold Greek
%
%%%%%%%%%%%%%%%%%%%%%%%%%%%%%%%%%%%%%%%%%%%%%%bar Bold Greek
%
\def\bbarbbupsilon{\bar\bbupsilon}
\def\bbarbbalpha{\bar\bbalpha}
\def\bbarbbbeta{\bar\bbbeta}
\def\bbarbbgamma{\bar\bbgamma}
\def\bbarbbdelta{\bar\bbdelta}
\def\bbarbbepsilon{\bar\bbepsilon}
\def\bbarbbvarepsilon{\bar\bbvarepsilon}
\def\bbarbbzeta{\bar\bbzeta}
\def\bbarbbeta{\bar\bbeta}
\def\bbarbbtheta{\bar\bbtheta}
\def\bbarbbvartheta{\bar\bbvartheta}

\def\bbarbbiota{\bar\bbiota}
\def\bbarbbkappa{\bar\bbkappa}
\def\bbarbblambda{\bar\bblambda}
\def\bbarbbmu{\bar\bbmu}
\def\bbarbbnu{\bar\bbnu}
\def\bbarbbxi{\bar\bbxi}
\def\bbarbbpi{\bar\bbpi}
\def\bbarbbrho{\bar\bbrho}
\def\bbarbbvarrho{\bar\bbvarrho}
\def\bbarbbvarsigma{\bar\bbvarsigma}
\def\bbarbbphi{\bar\bbphi}
\def\bbarbbvarphi{\bar\bbvarphi}
\def\bbarbbchi{\bar\bbchi}
\def\bbarbbpsi{\bar\bbpsi}
\def\bbarbbomega{\bar\bbomega}

\def\bbarbbGamma{\bar\bbGamma}
\def\bbarbbDelta{\bar\bbDelta}
\def\bbarbbTheta{\bar\bbTheta}
\def\bbarbbLambda{\bar\bbLambda}
\def\bbarbbXi{\bar\bbXi}
\def\bbarbbPi{\bar\bbPi}
\def\bbarbbSigma{\bar\bbSigma}
\def\bbarbbUpsilon{\bar\bbUpsilon}
\def\bbarbbPhi{\bar\bbPhi}
\def\bbarbbPsi{\bar\bbPsi}
%%%%%%%%%%%%%%%%%%%%%%%%%%%%%%%%%%%%%%%%%%%end of bar Bold Greek
%
%%%%%%%%%%%%%%%%%%%%%%%%%%%%%%%%%%%%%%%%%%%%%%hat Bold Greek
%
\def\hhatbbupsilon{\hat\bbupsilon}
\def\hhatbbalpha{\hat\bbalpha}
\def\hhatbbbeta{\hat\bbbeta}
\def\hhatbbgamma{\hat\bbgamma}
\def\hhatbbdelta{\hat\bbdelta}
\def\hhatbbepsilon{\hat\bbepsilon}
\def\hhatbbvarepsilon{\hat\bbvarepsilon}
\def\hhatbbzeta{\hat\bbzeta}
\def\hhatbbeta{\hat\bbeta}
\def\hhatbbtheta{\hat\bbtheta}
\def\hhatbbvartheta{\hat\bbvartheta}

\def\hhatbbiota{\hat\bbiota}
\def\hhatbbkappa{\hat\bbkappa}
\def\hhatbblambda{\hat\bblambda}
\def\hhatbbmu{\hat\bbmu}
\def\hhatbbnu{\hat\bbnu}
\def\hhatbbxi{\hat\bbxi}
\def\hhatbbpi{\hat\bbpi}
\def\hhatbbrho{\hat\bbrho}
\def\hhatbbvarrho{\hat\bbvarrho}
\def\hhatbbvarsigma{\hat\bbvarsigma}
\def\hhatbbphi{\hat\bbphi}
\def\hhatbbvarphi{\hat\bbvarphi}
\def\hhatbbchi{\hat\bbchi}
\def\hhatbbpsi{\hat\bbpsi}
\def\hhatbbomega{\hat\bbomega}

\def\hhatbbGamma{\hat\bbGamma}
\def\hhatbbDelta{\hat\bbDelta}
\def\hhatbbTheta{\hat\bbTheta}
\def\hhatbbLambda{\hat\bbLambda}
\def\hhatbbXi{\hat\bbXi}
\def\hhatbbPi{\hat\bbPi}
\def\hhatbbSigma{\hat\bbSigma}
\def\hhatbbUpsilon{\hat\bbUpsilon}
\def\hhatbbPhi{\hat\bbPhi}
\def\hhatbbPsi{\hat\bbPsi}
%%%%%%%%%%%%%%%%%%%%%%%%%%%%%%%%%%%%%%%%%%%end of hat Bold Greek
%
%%%%%%%%%%%%%%%%%%%%%%%%%%%%%%%%%%%%%%%%%%%%%%tilde Bold Greek
%
\def\tdbbupsilon{\tilde\bbupsilon}
\def\tdbbalpha{\tilde\bbalpha}
\def\tdbbbeta{\tilde\bbbeta}
\def\tdbbgamma{\tilde\bbgamma}
\def\tdbbdelta{\tilde\bbdelta}
\def\tdbbepsilon{\tilde\bbepsilon}
\def\tdbbvarepsilon{\tilde\bbvarepsilon}
\def\tdbbzeta{\tilde\bbzeta}
\def\tdbbeta{\tilde\bbeta}
\def\tdbbtheta{\tilde\bbtheta}
\def\tdbbvartheta{\tilde\bbvartheta}

\def\tdbbiota{\tilde\bbiota}
\def\tdbbkappa{\tilde\bbkappa}
\def\tdbblambda{\tilde\bblambda}
\def\tdbbmu{\tilde\bbmu}
\def\tdbbnu{\tilde\bbnu}
\def\tdbbxi{\tilde\bbxi}
\def\tdbbpi{\tilde\bbpi}
\def\tdbbrho{\tilde\bbrho}
\def\tdbbvarrho{\tilde\bbvarrho}
\def\tdbbvarsigma{\tilde\bbvarsigma}
\def\tdbbphi{\tilde\bbphi}
\def\tdbbvarphi{\tilde\bbvarphi}
\def\tdbbchi{\tilde\bbchi}
\def\tdbbpsi{\tilde\bbpsi}
\def\tdbbomega{\tilde\bbomega}

\def\tdbbGamma{\tilde\bbGamma}
\def\tdbbDelta{\tilde\bbDelta}
\def\tdbbTheta{\tilde\bbTheta}
\def\tdbbLambda{\tilde\bbLambda}
\def\tdbbXi{\tilde\bbXi}
\def\tdbbPi{\tilde\bbPi}
\def\tdbbSigma{\tilde\bbSigma}
\def\tdbbUpsilon{\tilde\bbUpsilon}
\def\tdbbPhi{\tilde\bbPhi}
\def\tdbbPsi{\tilde\bbPsi}
%%%%%%%%%%%%%%%%%%%%%%%%%%%%%%%%%%%%%%%%%%%end of tilde Bold Greek
\def\deltat{\triangle t}
%%%%%%%%%%%%%%%%%%%%%%%%%%%%%%%%%%%%%%%%%%%%%%hat greek
%
\def\hhattheta{\hat\theta}
%%%%%%%%%%%%%%%%%%%%%%%%%%%%%%%%%%%%%%%%%%%end of hat greek
%
%
%
%
\def\var{{\rm var}}
\def\bbtau{{\mbox{\boldmath $\tau$}}}
%
%
%
%
%
%misc
%\def\diag{{\text{diag}}}
%\def\tr{{\text{tr}}}
%\def\tdbbs{{\tilde{\bfs}}}
%\def\tdbby{{\tilde{\bfy}}}
%\def\rank{{\text{rank}}}
%\def\bbarbbs{{\bar{\bfs}}}
%\def\bbarbbH{{\bar{\bfH}}}
%\def\bbarbby{{\bar{\bfy}}}
%\def\tdbbH{{\tilde \bfH}}
%\def\tdbbh{{\tilde \bfh}}
%\def\tdbbeta{{\tilde \bfeta}}
%\def\checkbby{{\check\bfy}}
%\def\checkbbs{{\check\bfs}}
%\def\checkbbx{{\check\bfx}}
%\def\checkbbeta{{\check\bfeta}}
%\def\checktdbbs{{\check\tdbbs}}


\newcommand{\icomp}{\bar{i}}
\begin{document}


%%%%%%%%%%%%%%%%%%%%%%%%%%%%%%%%%%%%%%%%%%%%%%%%%%%%%%%%%%%%%%%%%%%%%
\title{Kernel VAR process estimation}
%%%%%%%%%%%%%%%%%%%%%%%%%%%%%%%%%%%%%%%%%%%%%%%%%%%%%%%%%%%%%%%%%%%%%

\if\reportmode1
  \author{Author\\[.5cm]\today}
\else
\author{Luis M. Lopez-Ramos,~\IEEEmembership{Member,~IEEE,}
\thanks{The work in this paper was supported by...
 }
\thanks{Author One and Author Two  are with the Dept. of ...,
  University of ..., City, PostalCode
  Country. E-mails:\{authorone,autortwo\}@university.domain.
Author Three is with the Dept. of ...,
  University of ..., City, PostalCode
  Country. E-mail:author.three@university.domain.}
\thanks{Parts of this work have been presented in Conference XYZ.}
}
\fi

\maketitle
\begin{abstract}
Kernel vector autoregressive model and least-square estimation with group-lasso regularization
\end{abstract}

\if\reportmode0
\begin{keywords}
Keyword 1, keyword 2, keyword 3, keyword 4
\end{keywords}
\fi

%\section{Introduction}
\begin{myitemize}
\item
\cmt{Modeling}
The most general nonlinear SVARM form can be written as:
  \begin{align}
\label{eq:svar:general}
	y_{j}[t]=\tilde{f}_j(\bby_{-j}[t], \{\bby[t-\ell]\}_{\ell=1}^L)+e_{j}[t],\quad j=1,\ldots, N
\end{align}
where $\bby_{-j}[t]$ denotes all entries of $\bby[t]$ except the $j$-th one, and $\bar{f}_j(.)$ denotes a nonlinear function of its multivariate argument.
\item \cmt{VAR only} We will assume purely VAR models here:
\begin{align}
\label{eq:svar:general}
	y_{j}[t]=\bar{f}_j(\{\bby[t-\ell]\}_{\ell=1}^L)+e_{j}[t],\quad j=1,\ldots, N
\end{align}
\item \cmt{Motivation for additive models}
The sought $\bar{f}$ functions entail $LN$ variables. The 'curse of dimensionality' for large $L$ and/or $N$ motivates introducing a model with simpler functions.
While \cite{shen2016nonlinear} proposes a \emph{generative additive model} where each $\bar{f}_j$ function is separable into a sum of $NP$ nonlinear univariate functions, we propose a model where each $\bar{f}_j$ is separable into a sum of $N$ nonlinear $P$-variate functions.
\item {\cmt{Model}
Our proposed nonlinear VAR model can be written as:
\begin{equation}
    y_{j}[t] = \sum_{i=1}^N f_{ij}(\bby_i[t]) + e_j[t]
\end{equation}
where $\bby_i[t] := \left[y_i[t-1], \ldots, y_i[t-P]\right]$, and $f_{ij}$ belongs to the parametric function space defined as
\begin{equation}
\label{def:parametric}
	\mathcal{H}_i:=\{f_{ij}|f_{ij}(\bby)=\sum_{d=1}^{\infty} \beta_{ij}^d{\varphi}^d(\bby); \beta_{ij}^d, \in \mathbb{R}\}.
\end{equation}
and $\{\varphi^d\}_{d=1}^\infty$ are nonlinear functions that map vectors in $\mathbb{R}^P$ into a feature space. The space $\mathcal{H}_i$ is equivalent to the RKHS defined as
\begin{equation}
\label{def:rkhs}
	\mathcal{H}_i:=\{f_{ij}|f_{ij}(\bby)=\sum_{\tau=1}^{\infty} \alpha_{ij\tau}\kappa_i(\bby, \bby_i[\tau]); \alpha_{ij\tau}\in \mathbb{R},\bby_i[\tau]\in\mathbb{R}^P  \}.
\end{equation}
}
\item \cmt{Criterion}
Our estimation criterion aims at enforcing the sparsity of edges. Since in this case each $f_{ij}$ is associated with each edge in the topology graph, the regularizing penalty is the number of edges that have a nonzero function. For a given node $j$:
\begin{equation}
    \{f^*_{ij}\}_{i=1}^N := \arg\min_{\{f_{ij} \in \mathcal{H}_i\}} \sum_t \left( y_j[t] - \sum_i f_{ij}(\bby_i[t]) \right)^2 + \lambda \sum_i \mathbb{1}\{\exists \bby: f_{ij}(\bby) \neq 0\}
\end{equation}
which equals the function parameterized by
\begin{equation}
    \{\beta_i^{d*}\}_{i=1}^N := \arg\min_{\{\beta_i^d\}} \sum_t \left( y_j[t] - \sum_i \sum_d \beta_i^d \varphi^d(\bby_i[t]) \right)^2 + \lambda \sum_i \mathbb{1}\{\| \boldsymbol{\beta}_i \|_2 \neq 0\}
\end{equation}
where $\boldsymbol{\beta}_i$ gathers the coefficients $\beta^d$ for $d=1, \ldots, \infty$.
Proceeding similarly to \cite{zaman2017online}, we get around the nonconvexity of this objective by replacing the indicator with its argument, resulting into its group-lasso surrogate. With $\boldsymbol{\phi}(\cdot)$ being a vector function whose image gathers $\phi^d(\cdot)$ for $d=1, \ldots, \infty$, we arrive at:
\begin{equation}
    \{\boldsymbol{\beta}^*_i\}_{i=1}^N := \min_{\{f_{ij} \in \mathcal{H}_i\}} \sum_t \left( y_j[t] - \sum_i \boldsymbol{\beta}_i^\top \boldsymbol\phi(\bby_i[t]) \right)^2 + \lambda \sum_i \| \boldsymbol{\beta}_i \|_2
\end{equation}
\item\cmt{Proposition} $\boldsymbol{\beta}_i = \boldsymbol{\Phi}_i\boldsymbol{\alpha}_i$ for some $\alpha_i$.
\end{myitemize}
%%%

%



%%%%%%%%%%%%% DO NOT MODIFY  %%%%%%%%%%%%%%%%%%%%%%%%%%%%%%
\if\editmode1 
\onecolumn
\printbibliography
\else
\bibliography{\bibfilenames}
\fi
\end{document}
%%%%%%%%%%%%%%%%%%%%%%%%%%%%%%%%%%%%%%%%%%%%%%%%%%%%%%%%%%%%


